%---------------------------------------------------------------
% SLO: slovenski povzetek
% ENG: slovenian abstract
%---------------------------------------------------------------
\selectlanguage{slovene} % Preklopi na slovenski jezik
\addcontentsline{toc}{chapter}{Povzetek}
\chapter*{Povzetek}

\noindent\textbf{Naslov:} \ttitle
\bigskip

Pomnilnik je kljub uvedbi pomnilniške hierarhije še vedno najpočasnejši del računalniške arhitekture, zato je upravljanje z njim ključnega pomena pri hitrosti izvajanja programov. V zadnjih letih se predvsem na področju sistemskega programiranja vedno bolj uveljavlja programski jezik Rust, ki za čiščenje pomnilnika uporablja princip lastništva in izposoje, s katerima zna med prevajanjem zagotoviti predvidljivo in učinkovito sproščanje pomnilnika. V obsegu magistrskega dela nas zanima, ali je mogoče princip lastništva implementirati v len funkcijski programski jezik STG, ki se uporablja pri prevajanju jezika Haskell. V magistrskem delu implementiramo simulator STG stroja, v katerega dodamo analizo lastništva. Pri tem se izkaže, da s tem zelo omejimo delovanje jezika, saj onemogočimo deljenje objektov. Predstavimo dve rešitvi: globoko kloniranje objektov, za katerega se izkaže, da je zelo časovno in prostorsko potratno, in model lastništva, ki zahteva analizo premikov in izposoj. A izkaže se, da analize izposoj ni mogoče implementirati brez uvedbe bodisi dodatnih informacij o izračunanosti objektov bodisi neke oblike neučakanosti v len programski jezik.

% in izposoje, pri katerem ugotovimo, da brez dodajanja dodatnih informacij o izračunanosti objektov analize izposoj ne moremo implementirati.

% Izkaže se, da analize izposoj v len programski jezik brez uvedbe neke oblike neučakanosti ne moremo implementirati.

% V vzorcu je predstavljen postopek priprave magistrskega dela z uporabo okolja \LaTeX. Vaš povzetek mora sicer vsebovati približno 100 besed, ta tukaj je odločno prekratek. Dober povzetek vključuje: (1) kratek opis obravnavanega problema, (2) kratek opis vašega pristopa za reševanje tega problema in (3) (najbolj uspešen) rezultat ali prispevek magistrske naloge.

\subsection*{Ključne besede}
\textit{\tkeywords}
\clearemptydoublepage