%================================================================
% SLO
%----------------------------------------------------------------
% datoteka: 	thesis_template.tex
%
% opis: 		predloga za pisanje diplomskega dela v formatu LaTeX na
% 				Univerza v Ljubljani, Fakulteti za računalništvo in informatiko
%
% pripravili: 	Matej Kristan, Zoran Bosnić, Andrej Čopar,
%			  	po začetni predlogi Gašperja Fijavža
%
% popravil: 	Domen Rački, Jaka Cikač, Matej Kristan
%
% verzija: 		30. september 2016 (dodan razširjeni povzetek)
%================================================================


%================================================================
% SLO: definiraj strukturo dokumenta
% ENG: define file structure
%================================================================
\documentclass[a4paper, 12pt]{book}
 

%================================================================
% SLO: Odkomentiraj "\SLOtrue " za izbiro slovenskega jezika
% ENG: Uncomment "\SLOfalse" to chose English languagge
%================================================================
\newif\ifSLO
\newif\ifTRACKEXIST
\newif\ifTRACKCS
\newif\ifPROGRAMMM

% ---------------------------------------------------------------------------------------
% IMPORTANT: Adjust the thesis language, your study program and course within this block
% ---------------------------------------------------------------------------------------
% switch language
\SLOtrue % Enables Slovenian language
% \SLOfalse  % Enables English language

% switch programs: Computer science and Multimedia. Set to false if the program is in Multimedia
\PROGRAMMMfalse
% \PROGRAMMMtrue

% switch on if your program is divided into tracks CS and DS, otherwise leave it false
% CAUTION: if you were first enrolled into your program before school year 2019/2020, your program is not divided into tracks. In any case, be absolutely sure you select the correct variant. IF IN DOUBT, always contact the student office to advise you.
%
% \TRACKEXISTfalse
\TRACKEXISTtrue

% default course name is "Computer science" if your course name is "Data science", set the following switch to false
\TRACKCStrue % uncomment if the thesis is from course "Information science"
%\TRACKCSfalse % uncomment if the thesis is from course "Data Science"
% -------------------------------------------------------------------------------------------
% End of language, program and course adjustment
% -------------------------------------------------------------------------------------------


%================================================================
% SLO: vključi oblikovanje in pakete
% ENG: include design and packages
%================================================================
%----------------------------------------------------------------
% SLO: LaTeX paketi
% ENG: LateX packages
%----------------------------------------------------------------
% SLO: omogoča uporabo slovenskih (latinskih) črk kodiranih v formatu UTF-8
% ENG: enables the use of slovene (latin) caracters encoded in the UFT-8 format
\usepackage[utf8x]{inputenc}
%\inputencoding{utf8}
% SLO: naloži, med drugim, slovenske delilne vzorce
% ENG: loads, among others, slovene dividing patterns
\usepackage[slovene,english]{babel}
% SLO: poskrbi za postavitev strani
% ENG: takes care of the page layout
\usepackage{fancyhdr}
% SLO: za vlaganje slik različnih formatov
% ENG: for loading figures of different formats
\usepackage{graphicx}
\usepackage{caption}
\captionsetup[figure]{labelfont=bf} % SLO: napis "Slika #" v krepkem tisku
									% ENG: wirte "Figure #" caption in bold
\captionsetup[table]{labelfont=bf} % SLO: napis "Tabela #" v krepkem tisku
								   % ENG: wirte "Table #" caption in bold
% SLO: za pisanje psevdokode
% ENG: for writing pseudocode
\usepackage{algorithm}
\usepackage{algorithmic}
\floatname{algorithm}{\footnotesize Algorithm} % SLO: napis "Algoritem #" v krepkem tisku
											   % ENG: write "Algorithm #" caption in bold
% SLO: poveže reference slik/tabel in slike/tabele znotraj dokumenta
% ENG: links image/table references with the images/tables within the document
\usepackage[pdfa]{hyperref}
% SLO: pri kliku na referenco slike/tabele se postavi na vrh slike/tabele
% ENG: when clicking the image/table reference, position the focus on top of the image/table
\usepackage[all]{hypcap}
% SLO: omogoča, med drugim, definicjo in uporebo barve
% ENG: enables, among others, the definition and use of colors
\usepackage{xcolor}
%----------------------------------------------------------------
% SLO: dodatni paketi
% ENG: additional packages
%----------------------------------------------------------------

% Pretvori celoten dokument v črno-belo
%\usepackage[gray]{xcolor}

% SLO: omogoča večjo manipulacijo nad tabelami
% ENG: allows for greater manipulation of tables
\usepackage{booktabs}
% SLO: naloži dodatne simbole
% ENG: loads additional symbols
\usepackage{amssymb}
% SLO: omogoča, med drugim, sklicevanje na formule z eqref
% ENG: enables, among others, equation referencing with eqref
\usepackage{amsmath}
\usepackage{mathtools}
% SLO: omogoča komentiranje večjega dela teksta
% ENG: enables the commenting of larger text parts
\usepackage{verbatim}
% SLO: omogoča rotacijo PDF strani v ležeč položaj
% ENG: enables the rotation of a PDF page to landscape
\usepackage{pdflscape}
% SLO: omogoča barvanje vrstic in stolpcev tabel
% ENG: enables coloring of table rows and columns
\usepackage{colortbl}
\usepackage{url}
\usepackage{tikz}
\usetikzlibrary{arrows}
\usetikzlibrary{tikzmark}
\usetikzlibrary{positioning,arrows.meta}
\usetikzlibrary{decorations.pathreplacing}
% Za shranjevanje tikz slik v zunanje datoteke.
% \usetikzlibrary{external}
% \tikzexternalize % activate!

\usepackage{subcaption}

% Minted package for code highlighting. To enable bold characters in code, the lmodern font family must also be included.
\usepackage{minted}
\usepackage{lmodern}

\usepackage{tcolorbox}
\tcbuselibrary{minted}
\tcbuselibrary{skins}

\usepackage{pifont}
\newcommand{\cmark}{\ding{51}}%
\newcommand{\xmark}{\ding{55}}%

\renewcommand{\theFancyVerbLine}{\sffamily
	\textcolor{lightgray}{\scriptsize
		\oldstylenums{\arabic{FancyVerbLine}}}}

\newtcblisting{myminted}[3][]{%
	listing engine=minted,
	minted options={
		linenos,
		breaklines,
		fontsize=\footnotesize,
		tabsize=4,
	},
	minted language=#2,
	listing only,
	colback=white,
	colframe=lightgray,
	enhanced,
	boxrule=0.5pt,
	sharp corners,
	left=2em,
	overlay={\node[anchor=south east, font=\small, outer sep=0pt,xshift=-3em,yshift=-0.5em,fill=white,inner xsep=0.75em] at (frame.south east) {#3};},
	#1
}

\newtcblisting{rust-success}[1][]{%
	listing engine=minted,
	minted options={
		linenos,
		breaklines,
		fontsize=\footnotesize,
		tabsize=4,
	},
	minted language=rust,
	listing only,
	colback=white,
	colframe=lightgray,
	enhanced,
	boxrule=0.5pt,
	sharp corners,
	left=2em,
	overlay={\node[anchor=south east, font=\small, outer sep=0pt,xshift=-3em,yshift=-0.5em,fill=white,inner xsep=0.75em] at (frame.south east) {Rust \cmark};},
	#1
}

\newtcblisting{rust-failure}[1][]{%
	listing engine=minted,
	minted options={
		linenos,
		breaklines,
		fontsize=\footnotesize,
		tabsize=4,
	},
	minted language=rust,
	listing only,
	colback=white,
	colframe=lightgray,
	enhanced,
	boxrule=0.5pt,
	sharp corners,
	left=2em,
	overlay={\node[anchor=south east, font=\small, outer sep=0pt,xshift=-3em,yshift=-0.5em,fill=white,inner xsep=0.75em] at (frame.south east) {Rust \xmark};},
	#1
}

\newtcblisting{code-box}[3][]{%
	listing engine=minted,
	minted options={
		linenos,
		breaklines,
		fontsize=\footnotesize,
		tabsize=4,
	},
	minted language=#2,
	listing only,
	colback=white,
	colframe=lightgray,
	enhanced,
	boxrule=0.5pt,
	sharp corners,
	left=2em,
	overlay={\node[anchor=south east, font=\small, outer sep=0pt,xshift=-3em,yshift=-0.5em,fill=white,inner xsep=0.75em] at (frame.south east) {#3};},
	#1
}

% Defines an environment which typesets code with language STG and style stgyle.
% Usage:
% \begin{stgcode}
% main = THUNK(1)
% \end{stgcode}
\newminted{stg}{style=stgyle, tabsize=2}
\newenvironment{env}{\VerbatimEnvironment\begin{stgcode}}{\end{stgcode}}

%================================================================
% SLO: nastavitve dokumenta
% ENG: document properties
%================================================================
% SLO: prilagoditev robov za tisk
% ENG: margin adjustments for printing
\addtolength{\marginparwidth}{-20pt}
\addtolength{\oddsidemargin}{40pt}
\addtolength{\evensidemargin}{-40pt}
% SLO: razmik med vrsticami
% ENG: line spacing
\renewcommand{\baselinestretch}{1.3}
% SLO: postavitev strani
% ENG: page layout
\renewcommand{\chaptermark}[1]{\markboth{\MakeUppercase{\thechapter.\ #1}}{}}
\renewcommand{\sectionmark}[1]{\markright{\MakeUppercase{\thesection.\ #1}}}
\renewcommand{\headrulewidth}{0.5pt} % Header rule
\renewcommand{\footrulewidth}{0pt} % Footer rule
%
\fancypagestyle{frontmatter}{%
	\fancyhf{} % Clear all headers and footers first
	\fancyhead[LE, RO]{\sl \thepage}
	%\fancyhead[LO]{\sl \rightmark}
	%\fancyhead[RE]{\sl \leftmark}
}
\fancypagestyle{mainmatter}{%
  	\fancyhf{} % Clear all headers and footers first
	\fancyhead[LE,RO]{\sl \thepage}
	\fancyhead[LO]{\sl \rightmark}
	\fancyhead[RE]{\sl \leftmark}
}
% SLO: font za ime avtorja
% ENG: font for author name
\newcommand{\authorfont}{\Large}
% SLO: font za naslov diplomskega dela
% ENG: font for thesis title
\newcommand{\titlefont}{\LARGE\bf}
% SLO: globina kazala
% ENG: content depth
\setcounter{tocdepth}{1}
% SLO: definiraj ukaz za prazno stran
% ENG: define the command for empty page
\newcommand{\clearemptydoublepage}{\newpage{\pagestyle{empty}\cleardoublepage}}

% course title
\newcommand{\trackname}{!undefined!}

\newcommand{\BibTeX}{{\sc Bib}\TeX}

\newcommand{\komentar}[1]{\textcolor{red}{#1}}

\newcommand{\infer}[4][]{
	\frac{#2}{#3 \enspace \Rightarrow \enspace #4}
	\if\relax\detokenize{#1}\relax
	\notag
	\else
	\tag{\textsc{#1}}
	\fi
}

\newcommand{\linfer}[4][]{
	\frac{#2}{\splitdfrac{#3}{\Rightarrow \enspace #4\hfill}}
	\if\relax\detokenize{#1}\relax
	\notag
	\else
	\tag{\textsc{#1}}
	\fi
}




\newcommand{\ttitle}{Lastništvo objektov namesto avtomatskega čistilca pomnilnika med lenim izračunom}
\newcommand{\ttitleEn}{Ownership model instead of garbage collection during lazy evaluation}
\newcommand{\tsubject}{\ttitle}
\newcommand{\tsubjectEn}{\ttitleEn}
\newcommand{\tauthor}{Niki Bizjak}
\newcommand{\temail}{nb2020@student.uni-lj.si}
\newcommand{\myyear}{2023}
\newcommand{\tkeywords}{prevajalnik, nestrog izračun, upravljanje s pomnilnikom, avtomatični čistilec pomnilnika, lastništvo objektov}
\newcommand{\tkeywordsEn}{compiler, lazy evaluation, memory management, garbage collector, ownership model}
\newcommand{\mysupervisor}{doc.~dr.\ Boštjan Slivnik}
\newcommand{\mycosupervisor}{}

% include formatted front pages
\input{style/thesis_front_pages}

%================================================================
% ENG: main pages of the thesis
%================================================================

%----------------------------------------------------------------
% Poglavje (Chapter) 1
%----------------------------------------------------------------
\chapter{Uvod}
\label{ch:uvod}

Pomnilnik je dandanes kljub uvedbi pomnilniške hierarhije še vedno eden izmed najpočasnejših delov računalniške arhitekture. Učinkovito upravljanje s pomnilnikom je torej ključnega pomena za učinkovito izvajanje programov. Upravljanje s pomnilnikom v grobem ločimo na ročno in avtomatično~\cite{jones2023garbage}. Pri ročnem upravljanju s pomnilnikom programski jezik vsebuje konstrukte za dodeljevanje in sproščanje pomnilnika. Odgovornost upravljanja s pomnilnikom leži na programerju, zato je ta metoda podvržena človeški napaki. Pogosti napaki sta puščanje pomnilnika (angl. memory leaking), pri kateri dodeljen pomnilnik ni sproščen, in viseči kazalci (angl. dangling pointers), ki kažejo na že sproščene in zato neveljavne dele pomnilnika~\cite{jones2023garbage}.

Pri avtomatičnem upravljanju s pomnilnikom zna sistem sam dodeljevati in sproščati pomnilnik. Tukaj ločimo posredne in neposredne metode. Ena izmed neposrednih metod je npr. štetje referenc~\cite{collins1960method}, pri kateri za vsak objekt na kopici hranimo metapodatek o številu kazalcev, ki se sklicujejo nanj. V tem primeru moramo ob vsakem spreminjanju referenc zagotavljati še ustrezno posodabljanje števcev, kadar pa število kazalcev pade na nič, objekt izbrišemo iz pomnilnika. Posredne metode, npr. označi in pometi~\cite{mccarthy1960recursive}, ne posodabljajo metapodatkov na pomnilniku ob vsaki spremembi, temveč se izvedejo le, kadar se prekorači velikost kopice. Algoritem pregleda kopico in ugotovi, na katere objekte ne kaže več noben kazalec ter jih odstrani. Nekateri algoritmi podatke na kopici tudi defragmentirajo in s tem zagotovijo boljšo lokalnost ter s tem boljše predpomnjenje~\cite{fenichel1969lisp}.

Avtomatično čiščenje pomnilnika pa ima tudi svoje probleme. Štetje referenc v primeru pomnilniških ciklov privede do puščanja pomnilnika, metoda označi in pometi pa nedeterministično zaustavi izvajanje glavnega programa in tako ni primerna za časovno-kritične (angl. real-time) aplikacije. Kot alternativa obem načinom upravljanja s pomnilnikom sistemski programski jezik Rust implementira model lastništva~\cite{klabnik2023rust}. Med \textit{prevajanjem} zna s posebnimi pravili zagotoviti, da se pomnilnik objektov na kopici avtomatično sprosti, kadar jih program več ne potrebuje. To pa zna storiti brez čistilca pomnilnika in brez eksplicitnega dodeljevanja in sproščanja pomnilnika, zato zagotavlja predvidljivo sproščanje pomnilnika.

V magistrskem delu se bomo primarno ukvarjali s STG jezikom. Operacijska semantika tega veleva, da so vsi izrazi v izvorni kodi v pomnilniku predstavljeni kot zaprtja. Jezik vsebuje izraz \texttt{let}, ki na kopici ustvari novo zaprtje, izbirni izraz \texttt{case} pa šele dejansko izračuna njegovo vrednost. Jezik STG za čiščenje zaprtij iz kopice uporablja generacijski čistilec pomnilnika~\cite{jones1992implementing, marlow2004making}. 

Cilj magistrske naloge je pripraviti simulator STG stroja, nato pa spremeniti STG jezik tako, da bo namesto avtomatičnega čistilca pomnilnika uporabljal model lastništva po zgledu programskega jezika Rust. Zanimalo nas bo, kakšne posledice to v STG stroj prinese, kakšne omejitve se pri tem pojavijo ter do kakšnih problemov lahko pri tem pride. Zavedati se moramo, da obstaja možnost, da koncepta lastništva ni mogoče vpeljati v STG stroj brez korenitih sprememb zasnove stroja samega - v tem primeru bomo podali analizo, zakaj lastništva v STG stroj ni mogoče vpeljati.

% POGLAVJE: PREGLED PODROČJA
\chapter{Pregled področja}
\label{ch:pregled-podrocja}

% Uvod v poglavje
\komentar{Mislim, da je trenutna vsebina tega poglavja to, kar bi o sorodnem delu želel povedati. Potrebno se mi zdi še napisati nek boljši uvod v poglavje.}

Da je upravljanje s pomnilnikom v funkcijskih programskih jezikih še vedno aktualno področje raziskovanja, priča mnogo eksperimentalnih jezikov razvitih v zadnjih nekaj letih. V sledečem poglavju bomo na kratko predstavili nekaj funkcijskih jezikov, ki za upravljanje s pomnilnikom uporabljajo manj konvencionalne pristope, nato pa se bomo podrobneje posvetili sistemom tipov in na kratko predstavili \textit{neučakan} programski jezik Granule, ki za upravljanje s pomnilnikom že uporablja princip lastništva in izposoje na podlagi tistega iz Rusta.

Ena izmed alternativ STG stroja za izvajanje jezikov z nestrogo semantiko je  prevajalnik GRIN~\cite{boquist1997grin} (angl. graph reduction intermediate notation), ki podobno kot STG stroj definira majhen funkcijski programski jezik, ki ga zna izvajati s pomočjo redukcije grafa. Napisane ima prednje dele za Haskell, Idris in Agdo, ponaša pa se tudi z zmožnostjo optimizacije celotnih programov (angl. whole program optimization)~\cite{podlovics2022modern}. Za upravljanje s pomnilnikom se v trenutni različici uporablja čistilec pomnilnika~\cite{boquist1999code}.

Na podlagi principa lastništva in izposoje iz Rusta je nastal len funkcijski programski jezik Blang~\cite{turk2022len}. Interpreter jezika zna pomnilnik za ovojnice izrazov in spremenljivk med izvajanjem samodejno sproščati brez uporabe čistilcev, zatakne pa se pri sproščanju funkcij in delnih aplikacij. \komentar{Potrebno je bolj podrobno opisati kako deluje Blang.}

Programski jezik micro-mitten~\cite{corbyn:practical-static-memory-management} je programski jezik, podoben Rustu, ki za upravljanje s pomnilnikom uporablja princip ASAP (angl. As Static As Possible)~\cite{proust2017asap}. Prevajalnik namesto principa lastništva izvede zaporedje analiz pretoka podatkov (angl. data-flow), namen katerih je aproksimirati statično živost spremenljivk na kopici. Pri tem prevajalnik ne postavi dodatnih omejitev za pisanje kode, kot jih poznamo npr. v Rustu, kjer mora programer za pisanje delujoče in učinkovite kode v vsakem trenutku vedeti, katera spremenljivka si objekt lasti in kakšna je njena življenjska doba. Metoda ASAP še ni dovolj raziskana in tako še ni primerna za produkcijske prevajalnike.

% Podpoglavja
\section{Upravljanje s pomnilnikom}

Pomnilnik je dandanes, kljub uvedbi pomnilniške hierarhije, še vedno eden izmed najpočasnejših delov računalniške arhitekture. Učinkovito upravljanje s pomnilnikom je torej ključnega pomena za učinkovito izvajanje programov. Upravljanje s pomnilnikom v grobem ločimo na ročno in avtomatično~\cite{jones2023garbage}. Pri ročnem upravljanju s pomnilnikom programski jezik vsebuje konstrukte za dodeljevanje in sproščanje pomnilnika. Odgovornost upravljanja s pomnilnikom leži na programerju, zato je ta metoda podvržena človeški napaki. Pogosti napaki sta puščanje pomnilnika (angl. memory leaking), pri kateri dodeljen pomnilnik ni sproščen in viseči kazalci (angl. dangling pointers), ki kažejo na že sproščene in zato neveljavne dele pomnilnika~\cite{jones2023garbage}.

Pri ročnem upravljanju pomnilnika, kot ga poznamo npr. pri programskem jeziku C, pride pogosto do dveh vrst napak~\cite{jones2023garbage}:

\begin{itemize}
	\itemsep 0em
	\item uporaba po sproščanju (angl. use-after-free), pri kateri program dostopa do bloka pomnilnika, ki je že bil sproščen in
	\item dvojno sproščanje (angl. double free), pri katerem se skuša isti blok pomnilnika sprostiti dvakrat.
\end{itemize}

V obeh primerih pride do nedefiniranega obnašanja sistema za upravljanje s pomnilnikom (angl. memory management system). Ob uporabi po sproščanju lahko pride npr. do dostopanja do pomnilniškega naslova, ki ni več v lasti trenutnega procesa in posledično do sesutja programa. V primeru dvojnega sproščanja pa lahko pride do okvare delovanja sistema za upravljanje s pomnilnikom, kar lahko privede do dodeljevanja napačnih naslovov ali prepisovanja obstoječih podatkov na pomnilniku.

Druga možnost upravljanja s pomnilnikom je avtomatično upravljanje pomnilnika, pri katerem zna sistem sam dodeljevati in sproščati pomnilnik. Pri avtomatičnem upravljanju s pomnilnikom nikoli ne pride do visečih kazalcev, saj je objekt na kopici odstranjen le v primeru, da nanj ne kaže kazalec iz nobenega drugega živega objekta. Ker pri je avtomatičnem upravljanju sistem za upravljanje s pomnilnikom edina komponenta, ki sprošča pomnilnik, je tudi zagotovljeno, da nikoli ne pride do dvojnega sproščanja. Glede na način delovanja ločimo posredne in neposredne metode. Pri neposrednih metodah zna sistem za upravljanje s pomnilnikom prepoznati živost objekta neposredno iz zapisa objekta na pomnilniku, pri posrednih metodah pa sistem s sledenjem kazalcem prepozna vse žive objekte, vse ostalo pa smatra za nedostopne oziroma neuporabljene objekte in jih ustrezno počisti.

Ena izmed neposrednih metod je npr. štetje referenc~\cite{collins1960method}, pri kateri za vsak objekt na kopici hranimo metapodatek o številu kazalcev, ki se sklicujejo nanj. V tem primeru moramo ob vsakem spreminjanju referenc zagotavljati še ustrezno posodabljanje števcev, kadar pa število kazalcev pade na nič, objekt izbrišemo iz pomnilnika. Ena izmed slabosti štetja referenc je, da mora sistem ob vsakem prirejanju vrednosti v spremenljivko posodobiti še števec v pomnilniku, kar privede do povečanja števila pomnilniških dostopov. Prav tako pa metoda ne deluje v primeru pomnilniških ciklov, kjer dva ali več objektov kažeta drug na drugega. V tem primeru števec referenc nikoli ne pade na nič, kar pomeni da pomnilnik nikoli ni ustrezno počiščen.

Posredne metode, npr. označi in pometi~\cite{mccarthy1960recursive}, ne posodabljajo metapodatkov na pomnilniku ob vsaki spremembi, temveč se izvedejo le, kadar se prekorači velikost kopice. Algoritem pregleda kopico in ugotovi, na katere objekte ne kaže več noben kazalec ter jih odstrani. Nekateri algoritmi podatke na kopici tudi defragmentirajo in s tem zagotovijo boljšo lokalnost ter s tem boljše predpomnjenje~\cite{fenichel1969lisp}. Problem metode označi in pometi pa je njeno nedeterministično izvajanje, saj programer oziroma sistem ne moreta predvideti, kdaj se bo izvajanje glavnega programa zaustavilo, in tako ni primerna za časovno-kritične (angl. real-time) aplikacije. Za razliko od štetja referenc pa zna algoritem označi in pometi počistiti tudi pomnilniške cikle in se ga zato včasih uporablja v kombinaciji s štetjem referenc. Programski jezik Python za čiščenje pomnilnika primarno uporablja štetje referenc, periodično pa se izvede še korak metode označi in pometi, da odstrani pomnilniške cikle, ki jih prva metoda ne zmore~\cite{van2007python}.
\section{Rustov model lastništva}

Programski jezik Rust je namenjen nizkonivojskemu programiranju, tj. programiranju sistemske programske opreme. Kot tak mora omogočati hitro in predvidljivo sproščanje pomnilnika, zato avtomatični čistilnik pomnilnika ne pride v poštev. Rust namesto tega implementira model lastništva~\cite{klabnik2023rust}, pri katerem zna med \textit{prevajanjem} s posebnimi pravili zagotoviti, da se pomnilnik objektov na kopici avtomatično sprosti, kadar jih program več ne potrebuje. Po hitrosti delovanja se tako lahko kosa s programskim jezikom C, pri tem pa zagotavlja varnejše upravljanje s pomnilnikom kot C.

Rust doseže varnost pri upravljanju pomnilnika s pomočjo principa izključitve~\cite{Jung}. V poljubnem trenutku za neko vrednost na pomnilniku velja natanko ena izmed dveh možnosti:

\begin{itemize}
	\itemsep 0em
	\item Vrednost lahko mutiramo preko \textit{natanko enega} unikatnega kazalca
	\item Vrednost lahko beremo preko poljubno mnogo kazalcev
\end{itemize}

\subsection{Premik}

Eden izmed najpomembnejših konceptov v Rustu je lastništvo. Ta zagotavlja, da si vsako vrednost na pomnilniku lasti natanko ena spremenljivka. Kadar gre ta izven dosega (angl. out-of-scope), lahko tako Rust ustrezno počisti pomnilnik. Ob klicu funkcije se lastništvo nad argumenti prenese v funkcijo, ta pa postane odgovorna za čiščenje pomnilnika. Naslednji primer prikazuje program, ki se v Rustu ne prevede zaradi težav z lastništvom.

\begin{rust-failure}
let number = Complex(0, 1);
let a = number;
let b = number;  // Napaka: use of moved value: `number`
\end{rust-failure}

Spremenljivka \texttt{a} prevzame lastništvo nad vrednostjo \texttt{number}. Pravimo tudi, da je bila vrednost \texttt{number} \textit{premaknjena} v spremenljivko \texttt{a}. Ob premiku postane spremenljivka \texttt{number} neveljavna, zato pri ponovnem premiku v spremenljivko \texttt{b}, prevajalnik javi napako.

\begin{rust-failure}
let number = Complex(1, 0);
f(number);
let x = number;  // Napaka: use of moved value: `number`
\end{rust-failure}

V zgornjem primeru se spremenljivka \texttt{number} pojavi kot argument funkciji \texttt{f}. Lastništvo spremenljivke \texttt{number} se ob klicu funkcije prenese v funkcijo in ta postane tudi odgovorna za čiščenje njenega pomnilnika. Ob ponovnem premiku v spremenljivko \texttt{x}, prevajalnik javi napako.

\subsection{Izposoja}

Drugi koncept, ki ga definira Rust je \textit{izposoja}. Ta omogoča \textit{deljenje} (angl. aliasing) vrednosti na pomnilniku. Izposoje so lahko spremenljive (angl. mutable) \texttt{\&mut x} ali nespremenljive (angl. immutable) \texttt{\&x}. V danem trenutku je lahko ena spremenljivka izposojena nespremenljivo oziroma samo za branje (angl. read-only) večkrat, spremenljivo pa natanko enkrat.

Naslednji primer se v Rustu uspešno prevede, ker sta obe izposoji nespremenljivi. Vrednosti spremenljivk \texttt{a} in \texttt{b} lahko le beremo, ne moremo pa jih spreminjati.

\begin{rust-success}
let number = Complex(2, 1);
let a = &number;
let b = &number;
\end{rust-success}

V naslednjem primeru skušamo ustvariti dve mutable referenci na spremenljivko \texttt{number}. Zaradi principa izključevanja to ni mogoče, zato prevajalnik javi napako.

\begin{rust-failure}
let mut number = Complex(1, 2);
let a = &mut number;
let b = &mut number;  // Napaka: cannot borrow `number` as mutable
                      // more than once at a time
\end{rust-failure}

Prav tako ni veljavno ustvariti mutabilne reference na spremenljivko, dokler nanjo obstaja kakršnakoli druga referenca.

\begin{rust-failure}
let mut number = Complex(0, 0);
let a = &number;
let b = &mut number;  // Napaka: cannot borrow `number` as mutable
                      // because it is also borrowed as immutable
\end{rust-failure}

\subsection{Delna izposoja}

\begin{rust-success}
let number = Complex(2, 1);
let a = number.1;
let b = number.2;
\end{rust-success}

\subsection{Ponovna izposoja}

\begin{rust-success}
let number = Complex(2, 1);
let a = &number;
let b = *a;
\end{rust-success}

% Non-lexical lifetimes
V trenutni različici Rusta 2018 bi se vsi zgornji primeri vseeno uspešno prevedli. Projekt Polonius~\cite{Matsakis_2018, Matsakis_et_al_2022} je v programski jezik namreč uvedel neleksikalne življenjske dobe (angl. non-lexical lifetimes), ki preverjanje izposoj in premikov izvaja nad CFG namesto nad abstraktnim sintaktičnim drevesom programa~\cite{Oxide, StackedBorrows}. Prevajalnik zna s pomočjo neleksikalnih življenskih dob dokazati, da sta dve zaporedni mutable izposoji varni, če ena izmed njih ni nikjer uporabljena. Na tak način Rust zagotovi bolj drobnozrnat (angl. fine grained) pogled na program, saj prevajalnik vrača napake ob manj veljavnih programih. Za vse zgornje primere torej predpostavljamo, da tako premaknjene, kot tudi izposojene spremenljivke spremenljivke v nadaljevanju še nekje uporabljene.

% Formalizacije Rusta
Kljub temu, da Rust v svoji dokumentaciji~\cite{klabnik2023rust} zagotavlja, da je njegov model upravljanja s pomnilnika varen, pa njegovi razvijalci niso nikoli uradno formalizirali niti njegove operacijske semantike, niti sistema tipov, niti modela za prekrivanje (angl. aliasing model). V literaturi se je tako neodvisno uveljavilo več modelov, ki skušajo čimbolj natančno formalizirati semantiko premikov in izposoj. Model Patina~\cite{reed2015patina} formalizira delovanje analize izposoj v Rustu in dokaže, da velja izrek o varnosti (angl. soundness) za safe podmnožico jezika Rust. Model RustBelt~\cite{10.1145/3158154} poda prvi formalen in strojno preverjen dokaz o varnosti za jezik, ki predstavlja realistično podmnožico jezika Rust. Model Stacked borrows~\cite{StackedBorrows} definira operacijsko semantiko za dostopanje do pomnilnika v Rustu in model za prekrivanje (angl. aliasing model) in ga strojno dokaže. Model Oxide~\cite{Oxide} je formalizacija Rustovega sistema tipov in je tudi prva semantika, ki podpira tudi neleksikalne življenjske dobe. 
\chapter{STG}
\label{ch:stg}

Na tem mestu bomo na kratko opisali delovanje prevajalnika GHC (angl. Glasgow Haskell compiler), ki izvorno kodo napisano v programskem jeziku Haskell prevede v strojno kodo. Pri prevajanju se program transformira v več različnih vmesnih predstavitev (angl. intermediate representation), mi pa se bomo v magistrskem delu osredotočili predvsem na vmesno kodo imenovano STG jezik (angl. Spineless tagless G-Machine language), katerega delovanje bomo podrobneje opisali v razdelku \ref{sec:stg-jezik}.

\subsection{Prevajalnik GHC}
\label{sec:prevajalnik-ghc}

% TODO: Dodaj shemo prevajanja Haskell -> strojna koda

Prevajalnik GHC (angl. Glasgow Haskell compiler) prevajanje iz izvorne kode v programskem jeziku Haskell v strojno kodo izvaja v več zaporednih fazah oziroma modulih. Vsaka faza kot vhod prejme izhod prejšnje, nad njim izvede določeno transformacijo in rezultat posreduje naslednji fazi. Faze glede na njihovo funkcijo v grobem delimo na tri dele. V prednjem delu (angl. front-end) se nad izvorno kodo najprej izvede leksikalna analiza, pri kateri se iz toka znakov, ki predstavljajo vhodni program pridobi abstraktno sintaktično drevo (angl. abstract syntax tree). Nad drevesom se izvede še zaporedje semantičnih analiz pri katerih se preveri ali je program pomensko pravilen. Sem sodi razreševanje imen, pri kateri se razreši vsa imena spremenljivk iz uvoženih modulov v programu in preveri ali so vse spremenljivke deklarirane pred njihovo uporabo. Izvede se še preverjanje tipov, kjer se za vsak izraz izpelje njegov najbolj splošen tip in preveri ali se vsi tipi v programu ujemajo.

\begin{figure}[h]
	\centering
	\begin{tikzpicture}
		\tikzset{
			every node/.append style={text width=1.4cm,execute at begin node=\setlength{\baselineskip}{1em},font=\footnotesize},
			block/.style={draw,rectangle,text width=2cm,align=center,minimum height=1cm,minimum width=2cm},
		}
		
		% Srednji del prevajalnika
		\node[block] (optimizacija) {Optimizacija};
		\node[coordinate, left=2.6cm of optimizacija] (center-levo) {};
		\node[coordinate, right=2.6cm of optimizacija] (center-desno) {};
		
		% Prednji del prevajalnika
		\node[block, above=0.55cm of center-levo] (semanticna-analiza) {Semantična analiza};
		\node[block, above=0.55cm of semanticna-analiza] (razclenjevanje) {Razčlen\-je\-van\-je};
		\node[block, below=0.55cm of semanticna-analiza] (izpeljava-tipov) {Izpeljava tipov};
		\node[block, below=0.55cm of izpeljava-tipov] (razsladenje) {Razsladenje};
		
		% Zadnji del prevajalnika
		\node[block, above=0.55cm of center-desno] (stg-to-cmm) {Izbira \texttt{C--} ukazov};
		\node[block, above=0.55cm of stg-to-cmm] (core-to-stg) {Izbira STG ukazov};
		
		% Moznosti prevajanja C-- -> strojna koda
		\node[block, minimum height=0.6cm, below=0.9cm of stg-to-cmm] (neposredno) {Neposredno};
		\node[block, minimum height=0.6cm, below=0.3cm of neposredno] (gcc) {GCC};
		\node[block, minimum height=0.6cm, below=0.3cm of gcc] (llvm) {LLVM};
		
		% 
		\node[coordinate, left=0.4cm of neposredno.west] (levo-od-neposredno) {};
		\node[coordinate, left=0.4cm of gcc.west] (levo-od-gcc) {};
		\node[coordinate, left=0.4cm of llvm.west] (levo-od-llvm) {};
		\draw[-] (levo-od-neposredno) -- (levo-od-gcc) -- (levo-od-llvm);
		\draw[->] (levo-od-neposredno) -- (neposredno);
		\draw[->] (levo-od-gcc) -- (gcc);
		\draw[->] (levo-od-llvm) -- (llvm);
		
		\node[coordinate, right=0.4cm of neposredno.east] (desno-od-neposredno) {};
		\node[coordinate, right=0.4cm of gcc.east] (desno-od-gcc) {};
		\node[coordinate, right=0.4cm of llvm.east] (desno-od-llvm) {};
		\draw[-] (desno-od-neposredno) -- (desno-od-gcc) -- (desno-od-llvm);
		\draw[-] (neposredno) -- (desno-od-neposredno);
		\draw[-] (gcc) -- (desno-od-gcc);
		\draw[-] (llvm) -- (desno-od-llvm);
		
		\node[coordinate, below=0.55cm of stg-to-cmm] (pod-stg-to-cmm) {};
		\node[coordinate] at (pod-stg-to-cmm -| levo-od-neposredno) (nad-levo-od-neposredno) {};
		\draw[-] (stg-to-cmm) -- (pod-stg-to-cmm) node[pos=0.5,right] {\scriptsize \texttt{C--}} -- (nad-levo-od-neposredno) -- (levo-od-neposredno) ;
		
		\draw[->] (razclenjevanje) -- (semanticna-analiza) node[pos=0.5,right] {\scriptsize AST};
		\draw[->] (semanticna-analiza) -- (izpeljava-tipov) node[pos=0.5,right] {\scriptsize AST};
		\draw[->] (izpeljava-tipov) -- (razsladenje) node[pos=0.5,right] {\scriptsize AST};
		
		\draw[->] (core-to-stg) -- (stg-to-cmm) node[pos=0.5,right] {\scriptsize STG};
		
		% Puscica "Generator vmesne kode" -> "Optimizacija vmesne kode"
		\node[coordinate, left=0.5cm of optimizacija.west] (levo-od-optimizacija) {};
		\node[coordinate] at (razsladenje -| levo-od-optimizacija) (desno-od-razsladenje) {};
		
		\draw[->] (razsladenje.east) -- (desno-od-razsladenje) -- (levo-od-optimizacija) node[pos=0.5,right] {\scriptsize Core} -- (optimizacija);
		
		% Puscica "Optimizacija vmesne kode" -> "Izbira ukazov"
		\node[coordinate, right=0.5cm of optimizacija.east] (desno-od-optimizacija) {};
		\node[coordinate] at (core-to-stg -| desno-od-optimizacija) (levo-od-izbira-ukazov) {};
		
		\draw[->] (optimizacija) -- (desno-od-optimizacija) -- (levo-od-izbira-ukazov) node[pos=1.0,left,align=right] {\scriptsize Core} -- (core-to-stg);
		
		% Vhodna povezava "Zacetek" -> "Leksikalna analiza"
		\node[coordinate, left=2cm of center-levo] (zacetek) {};
		\node[coordinate, left=0.5cm of razclenjevanje] (levo-od-razclenjevanje) {};
		\node[coordinate] at (zacetek -| levo-od-razclenjevanje) (desno-od-zacetek) {};
		
		\draw[->] (zacetek) -- (desno-od-zacetek) -- (levo-od-razclenjevanje) node[pos=0.2,left,text width=0.9cm] {\scriptsize tok\\znakov} -- (razclenjevanje);
		
		% Koncna povezava
		\node[coordinate, right=2cm of center-desno] (konec) {};
		\node[coordinate] at (konec -| desno-od-neposredno) (levo-od-konec) {};
		\draw[->] (desno-od-neposredno) -- (levo-od-konec) -- (konec) node[pos=1,above] {\scriptsize strojna\\koda};
		
		% Locevalne crte
		\node[coordinate, left=0.25cm of levo-od-optimizacija] (sprednji-del-center) {};
		\node[coordinate, above=3.7cm of sprednji-del-center] (sprednji-del-zgoraj) {};
		\node[coordinate, below=3.2cm of sprednji-del-center] (sprednji-del-spodaj) {};
		\draw[dashed] (sprednji-del-zgoraj) -- (sprednji-del-spodaj) node[pos=0,left=0.2cm,text width=2cm,align=right] {prednji del} node[pos=0,right=0.2cm,text width=2cm] {srednji del};
		
		\node[coordinate, right=0.25cm of desno-od-optimizacija] (zadnji-del-center) {};
		\node[coordinate, above=3.7cm of zadnji-del-center] (zadnji-del-zgoraj) {};
		\node[coordinate, below=3.2cm of zadnji-del-center] (zadnji-del-spodaj) {};
		\draw[dashed] (zadnji-del-zgoraj) -- (zadnji-del-spodaj) node[pos=0,right=0.2cm,text width=2cm] {zadnji del};
		
	\end{tikzpicture}
	\caption{Pomembnejše faze prevajalnika Glasgow Haskell compiler}
	\label{fig:shema-ghc}
\end{figure}

Ker je Haskell programski jezik, ki je namenjen ljudem, je precej velik jezik, z veliko različnimi sintaktičnimi oblikami in konstruktorji. Tako lahko npr. programer napiše isto kodo na več različnih načinov. S perspektive piscev prevajalnikov pa to pomeni veliko več dela, saj je potrebno zagotoviti, da se vsi konstrukti v strojno kodo prevedejo pravilno. Za odpravljanje te težave se v zadnjem koraku prednjega dela prevajalnika imenovanem razsladenje (angl. desugarification), sintaktično drevo jezika Haskell pretvori v sintaktično drevo jezika Core. Ta je zelo majhen funkcijski programski jezik, ki temelji na lambda računu, kjub svojem majhnem naboru konstruktov pa še vedno omogoča zapis poljubnega Haskell programa. Naslednje faze tako operirajo nad dosti manjšim jezikom, kar precej poenostavi prevajanje.

Srednji del (angl. middle-end) prevajalnika sestavlja zaporedje optimizacij, ki kot vhod sprejmejo program v Core jeziku in vrnejo izboljšan program v Core jeziku. Rezultat niza optimizacij se posreduje zadnjemu delu (angl. back-end) prevajalnika, ki poskrbi za prevajanje Core jezika v strojno kodo, ki se lahko neposredno izvaja na procesorju. Na tem mestu se Core jezik prevede v STG jezik, ta pa se nato prevede v programski jezik \texttt{C--}. Slednji je podmnožica programskega jezika C in ga je mogoče v strojno kodo prevesti na tri načine: neposredno ali z enim izmed prevajalnikov LLVM ali GCC. Prednost take vrste prevajanja je v večji prenosljivosti programov, saj znata LLVM in GCC generirati kodo za večino obstoječih procesorskih arhitektur, poleg tega pa imata vgrajene še optimizacije, ki pohitrijo delovanje izhodnega programa.

\subsection{STG jezik}
\label{sec:stg-jezik}

Lene funkcijske programske jezike najpogosteje implementiramo s pomočjo redukcije gra\-fa~\cite{peyton1987implementation}. Eden izmed načinov za izvajanje redukcije je abstraktni STG stroj (angl. Spineless Tagless G-machine)~\cite{jones1992implementing}, ki definira in zna izvajati majhen funkcijski programski jezik STG. STG stroj in jezik se uporabljata kot vmesni korak pri prevajanju najpopularnejšega lenega jezika Haskell v prevajalniku GHC (Glasgow Haskell Compiler)~\cite{GHC}.

% Kaj sploh je STG stroj in kaj je STG jezik

\subsubsection{Lena evalvacija}

Ena izmed najpomembnejših lastnosti lenih funkcijskih programskih jezikov je njihova nestroga semantika (angl. non-strict semantic). Ta narekuje kako se evalvirajo oziroma računajo argumenti pri klicu funkcije. Pri jezikih z nestrogo semantiko se računanje vrednosti argumentov funkcije ne izvede ob klicu, temveč šele takrat ko se izvaja telo funkcije. Ker se vrednosti argumentov ne izračunajo nemudoma, temveč se računanje zamakne v času dokler funkcija vrednosti arugmenta dejansko ne potrebuje, je taka semantika imenovana tudi za leno.

% TODO: Preveri kako deluje nestroga semantika v knjigi
$$ f \bot = \bot $$

Leno evalvacijo najpogosteje implementiramo s pomočjo zakasnitev (angl. thunks). Te so na pomnilniku predstavljene kot kazalec na kodo, ki izračuna njihovo vrednost. Ob evalvaciji zakasnitve se najprej izračuna njihova vrendost, izračunano vrednost pa se shrani v struktro na pomnilniku, da je ob naslednji evalvaciji ni potrebno ponovno računati. Na tak način se z nestrogo semantiko doseže, da se vsak izraz izračuna \textit{največ enkrat}. Če se argument ne pojavi nikjer v telesu funkcije, se zakasnitve nikoli ne računa, če pa se v telesu pojavi večkrat, se vrednost izračuna enkrat, za vsako nadaljno evalvacijo argumenta pa se preprosto vrne vrednost shranjeno na pomnilniku.

% TODO: Kaj je razlika med eval/apply in push/enter modelom
% Kaj se spremeni, če dodamo enega in drugega
% Katero različico trenutno uporablja Haskell

Nestrogo semantiko je moč implementirati na dva različna načina:

\begin{itemize}
	\itemsep 0em
	\item model potisni in vstopi (angl. push / enter model)
	\item model izračunaj in apliciraj (angl. eval / apply model)
\end{itemize}

Abstraktni STG stroj je zgodovinsko najprej uporabljal model potisni / vstopi~\cite{jones1992implementing}. Trenutna različica STG stroja pa uporablja model evalviraj / apliciraj~\cite{marlow2004making}, saj se je empirično izkazal za bolj učinkovito implementacijo nestroge semantike.

\subsubsection{Definicija jezika}

V sledečem poglavju bomo podali formalno definicijo jezika STG definirano v \cite{marlow2004making}. Od originalne implementacije v \cite{jones1992implementing} se razlikuje po tem, da implementacija ni več brez oznak (angl. tagless), temveč nosi vsak objekt na kopici še dodatno polje z informacijo o njegovi vrsti. Ker je število tipov objektov majhno, se za oznako objekta navadno uporablja kar celoštevilčna vrednost. V originalni implementaciji so bili vsi objekti na kopici predstavljeni enotno, kar je pomenilo bolj kompaktno predstavitev podatkov v pomnilniku, prav tako pa STG stroju ni bilo treba preverjati vrste objektov ob vsakem klicu funkcije. Prednost ponovne uvedbe oznak pa je v tem, da STG stroju ni potrebno vzdrževati dveh ločenih skladov za argumente in vrednosti, kar pa tudi poenostavi delovanje čistilca pomnilnika.
% TODO: Citat

Sledi formalna definicija STG jezika. Pri tem bomo spremenljivke označevali s poševnimi malimi tiskanimi črkami $x, y, f, g$, konstruktorje pa s poševnimi velikimi tiskanimi črkami $C$.
\begin{align*}
	literal \quad \coloneq& \quad \underline{int} \enspace \vert \enspace \underline{double} & \text{primitivne vrednosti}
\end{align*}

% TODO: Kakšen je prevod za boxed? Unboxed?
STG jezik podpira dva primitivna (angl. unboxed) podatkovna tipa: celoštevilske vrednosti in števila s plavajočo vejico. Poleg tega omogoča uvajanje novih algebraičnih podatkovnih tipov. Objekte algebraičnih tipov tvorimo s pomočjo konstruktorjev $C$.
\begin{align*}
	a, v \quad \coloneq& \quad literal \enspace \vert \enspace x & \text{argumenti so atomarni}
\end{align*}

Vsi argumenti pri aplikaciji funkcij in primitivnih operacij so v A-normalni obliki (angl. A-normal form)~\cite{flanagan1993essence}, kar pomeni, da so atomarni (angl. atomic). Tako je vsak argument ali primitivni podatkovni tip ali pa spremenljivka. Pri prevajanju v STG jezik lahko prevajalnik sestavljene argumente funkcij priredi novim spremenljivkam z ovijanjem v \texttt{let} izraz in spremenljivke uporabi kot argumente pri klicu funkcije. Pri tem je potrebno zagotoviti, da so definirane spremenljivke unikatne oziroma, da se ne pojavijo v ovitem izrazu. Aplikacijo funkcije $f \; (\oplus \; x \; y)$ bi tako ovili v \texttt{let} izraz $\texttt{let} \enspace a = \oplus \; x \; y \enspace \texttt{in} \enspace f \enspace a$, s čemer bi zagotovili, da so vsi argumenti atomarni.

\begin{align*}
	k \quad \coloneq& \quad \bullet & \text{neznana mestnost funkcije}\\
	\vert& \quad n & \text{znana mestnost $n \geq 1$}\\
\end{align*}

Prevajalnik lahko med prevajanjem za določene funkcije določi njihovo mestnost (angl. arity), tj. število argumentov, ki jih funkcija sprejme. Ker pa je STG funkcijski jezik, lahko funkcije nastopajo tudi kot argumenti drugih funkcij, zato včasih določevanje mestnosti ni mogoče. Povsem veljavno bi bilo vse funkcije v programu označiti z neznano mestnostjo $\bullet$, a je mogoče s podatkom o mestnosti klice funkcij implementirati bolj učinkovito, zato se med prevajanjem izvaja tudi analiza mestnosti. 

\begin{align*}
	expr \quad \coloneq& \quad a & \text{atom}\\
	\vert& \quad f^k a_1 \dots a_n & \text{klic funkcije ($n \geq 1$)}\\
	\vert& \quad \oplus a_1 \dots a_n & \text{primitivna operacija ($n \geq 1$)}\\
	\vert& \quad \texttt{let} \enspace x = obj \enspace \texttt{in} \enspace e & \text{} \\
	\vert& \quad \texttt{case} \enspace e \enspace \texttt{of} \enspace \{ alt_1; \dots; alt_n \}& \text{} \\
\end{align*}

% TODO: Prevedi saturated / saturirane v zasičene

Pri tem velja, da so vse primitivne operacije \textit{zasičene}, kar pomeni, da sprejmejo natanko toliko argumentov, kot je mestnost (angl. arity) funkcije. Če programski jezik omogoča delno aplikacijo primitivnih funkcij, potem je potrebno take delne aplikacije z $\eta$-dopolnjevanjem razširiti v saturirano obliko. Pri tem delno aplikacijo ovijemo v nove lambda izraze z uvedbo novih spremenljivk, ki se ne pojavijo nikjer v izrazu. Tako npr. izraz \texttt{(+ 3)}, ki predstavlja delno aplikacijo vgrajene funkcije za seštevanje prevedemo v funkcijo $\lambda x . (+ 3 x)$ in s tem zadostimo pogoju saturiranosti.

\begin{align*}
	alt \quad \coloneq& \quad C \enspace x_1 \dots x_n \to expr & \text{algebraična alternativa}\\
	\vert& \quad x \to expr & \text{privzeta alternativa}\\
\end{align*}

Podatkovne objekte na kopici je mogoče ustvarjati le z enim konstruktom, in sicer \texttt{let} izrazom. Ta nam omogoča prirejanje objekta spremeljivki, ki je vidna v telesu \texttt{let} izraza.

% TODO: Ali dodamo tudi naslednje?
% Dodamo najbrž šele potem, ko bomo govorili o konsturktorjih.
% Pomembno se nam zdi še poudariti, da sta pojma oznake objekta na kopici in oznake konstruktorja 

\begin{align*}
	obj \quad \coloneq& \quad \text{FUN}(x_1 \dots x_n \to e) & \text{aplikacija}\\
	\vert& \quad \text{PAP}(f \; a_1 \dots a_n) & \text{delna aplikacija}\\
	\vert& \quad \text{CON}(C \; a_1 \dots a_n) & \text{konstruktor}\\
	\vert& \quad \text{THUNK} \enspace e & \text{zakasnitev}\\
	\vert& \quad \text{BLACKHOLE} & \text{črna luknja}
\end{align*}

Objekt \textsc{FUN} predstavlja funkcijsko ovojnico (angl. closure) z argumenti $x_1, \dots, x_n$ in telesom $e$, ki pa se lahko poleg argumentov $x_i$ sklicuje še na druge proste spremenljivke. Pri tem velja, da je lahko funkcija aplicirana na več kot $n$ ali manj kot $n$ argumentov, tj. je curryrana.
% Andrej Bauer pravi, da je to okej.
% https://twitter.com/andrejbauer/status/621602561368399872

Objekt \textsc{PAP} predstavlja delno aplikacijo (angl. partial application) funkcije $f$ na argumente $x_1, \dots, x_n$. Pri tem je zagotovljeno, da bo $f$ objekt tipa \textsc{FUN}, katerega mestnost bo \textit{vsaj} $n$.

Objekt \textsc{CON} predstavlja saturirano aplikacijo konstruktorja $C$ na argumente $a1, \dots a_n$. Pri tem je število argumentov, ki jih prejme konstruktor natančno enako številu parametrov, ki jih zahteva.

Objekt \textsc{THUNK} predstavlja zakasnitev izraza $e$. Kadar se vrednost izraza uporabi, tj. kadar se izvede \texttt{case} izraz, se izračuna vrednost $e$, \textsc{THUNK} objekt na kopici pa se nato posodobi s preusmeritvijo (angl. indirection) na vrednost $e$. Pri evalvaciji zakasnitve se objekt \textsc{THUNK} na kopici zamenja z objektom \textsc{BLACKHOLE}, s čemer se preprečuje puščanje pomnilnika~\cite{jones1992tail} in neskončnih rekurzivnih struktur. Objekt \textsc{BLACKHOLE} se lahko pojavi le kot rezultat evalvacije zakasnitve, nikoli pa v vezavi v \texttt{let} izrazu.
\section{Sistemi tipov}
\label{sec:sistemi-tipov}

% Kratek opis sistemov tipov
Sistemi tipov formalno definirajo pravila za določanje podatkovnega tipa poljubnega izraza v programu. Poleg tega postavljajo omejitve, ki jih morajo vsi izrazi v programu izpolnjevati. Med njimi določijo, katere operacije se lahko izvajajo nad izrazi z določenimi tipi in kakšnega tipa je izračunan rezultat~\cite{pierce2002types}. Tipični primer tovrstnih pravil je, da se aritmetične operacije, kot sta seštevanje in množenje, lahko izvajajo le nad numeričnimi tipi, medtem ko je seštevanje nizov in števil prepovedano.

Glede na način preverjanja tipov ločimo statično in dinamično tipiziranje~\cite{pierce2002types}. Pri jezikih s statičnim tipiziranjem (angl. static typing) se podatkovni tipi izpeljejo oziroma izračunajo med prevajanjem, medtem ko se pri dinamičnem tipiziranju (angl. dynamic typing) preverjanje tipov izvaja med samim izvajanjem programa. Tako je pri statično tipiziranih jezikih, kot sta Java in Haskell, zagotovljeno, da se napake, povezane z nezdružljivostjo tipov, odkrijejo že med prevajanjem in ne med izvajanjem programa. Na drugi strani so jeziki z dinamičnim tipiziranjem, kot sta Python in JavaScript, podvrženi večjemu tveganju za napake med izvajanjem.

% STG ni tipiziran (Haskell pa je)
V poglavju \ref{ch:stg} bomo lahko videli, da STG jezik ne vključuje preverjanja tipov, vendar to ne pomeni, da lahko pride do težav zaradi tipov. Podatkovni tipi so namreč izpeljani in preverjeni pred prevajanjem Haskella v STG. Izrek o varnosti sistemov tipov Haskella zagotavlja, da \textit{med izvajanjem} STG jezika z redukcijo grafa ne bo prišlo do napak, povezanih z nezdružljivostjo tipov.

% Na kratko o izpeljavi / preverjanju tipov
V splošnem pri izpeljavi oziroma preverjanju tipov v kontekstu $\Gamma$ hranimo predpostavke o tipih spremenljivk, na katere smo že naleteli. Te predpostavke pomagajo pri določitvi tipov kompleksnejših izrazov in funkcij, saj lahko z uporabo že znanih tipov preverjamo in izpeljujemo tipe novih izrazov. Na ta način lahko v vsakem koraku preverimo, ali so tipi skladni s pravili sistema tipov.

% Substrukturni sistemi tipov
Večina programskih jezikov ima običajno neomejen (angl. unrestricted) sistem tipov, ki omogoča, da lahko do spremenljivk dostopamo poljubno mnogokrat in v poljubnem vrstnem redu. To je zagotovljeno s pomočjo treh strukturnih lastnosti~\cite{pierce2004advanced}:
\begin{itemize}
    \itemsep 0em
    \item \textit{Zamenjava} (angl. exchange) zagotavlja, da vrstni red spremenljivk v kontekstu tipov ni pomemben. Če je preverjanje tipov uspešno v nekem kontekstu $\Gamma$, potem bo uspešno tudi v kateremkoli drugem kontekstu, ki je sestavljen kot permutacija predpostavk iz $\Gamma$.
    \item \textit{Oslabitev} (angl. weakening) zagotavlja, da se lahko kljub dodajanju neuporabnih predpostavk v kontekst tipov izrazu še vedno določi tip.
    \item \textit{Zoženje} (angl. contraction) zagotavlja, da če lahko preverimo tip izraza z uporabo dveh enakih predpostavk, lahko isti izraz preverimo tudi z uporabo samo ene predpostavke. To pomeni, da lahko spremenljivko v izrazu uporabimo večkrat, ne da bi morali v kontekstu imeti več kopij te predpostavke.
\end{itemize}

Substrukturni sistemi tipov (angl. substructural type systems) ~\cite{pierce2004advanced} so sistemi tipov, pri katerih vsaj ena izmed treh strukturnih lastnosti ne velja. Mednje sodi npr. \textit{urejen} (angl. ordered) sistem tipov, pri katerem ne velja nobena izmed treh lastnosti, kar v praksi pomeni, da morajo biti vse spremenljivke uporabljene natanko enkrat in to v vrstnem redu, kot so bile deklarirane. V nadaljevanju si bomo podrobneje ogledali linearen sistem tipov, ki dovoljuje zamenjavo, ne pa tudi oslabitve in zoženja.

\section{Linearni tipi}
\label{sec:linearni-tipi}

% Kratek uvod v linearen sistem tipov
Linearni sistem tipov (angl. linear type system) je substrukturni sistem tipov, ki se od urejenega razlikuje v tem, da dovoljuje \textit{zamenjavo}~\cite{pierce2004advanced}. To pomeni, da zahteva, da je vsak objekt uporabljen \textit{natanko enkrat}, ne velja pa, da morajo biti ti uporabljeni v vrstnem redu, kot so bili deklarirani. Jeziki s sistemom linearnih tipov omogočajo pisanje programov, ki bolj varno upravljajo s pomnilnikom, saj zagotavljajo, da bo na vsako vrednost kazal natanko en kazalec, kar pomeni, da ne more priti do nenadzorovanih ali nepričakovanih sprememb podatkov, tudi če se program izvaja hkrati na več nitih. Prav tako pa omogočajo bolj natančen nadzor nad življenjskimi cikli spremenljivk, kar preprečuje pisanje programov, ki puščajo pomnilnik, ali programov, ki ne sprostijo virov operacijskega sistema po uporabi~\cite{pierce2004advanced}.

% Bolj formalna definicija linearnosti
Pri linearnih sistemih tipov gre referenca izven dosega takoj, ko se pojavi na desni strani prirejanja ali ko je posredovana kot argument funkciji. S tem je zagotovljeno, da na en objekt na kopici vedno kaže natanko ena referenca. Funkcija $f$ je linearna, če \textit{uporabi} svoj argument natanko enkrat. Linearno funkcijo $f$, ki kot vhod sprejme argument tipa $\alpha$ in vrne rezultat tipa $\beta$, označimo z oznako $f :: \alpha \multimap \beta$ .

Če sistem tipov ne omogoča oslabitve, potem je zagotovljeno, da nobene vrednosti ne moremo zavreči~\cite{wadler1991there}. V tem primeru leni izračun sploh ni potreben, saj bo vsaka vrednost zagotovo vsaj enkrat uporabljena. Če prepovemo pravilo zoženja, potem v jeziku vrednosti ne moremo podvajati, kar pomeni, da na vsako vrednost v pomnilniku kaže natanko ena referenca, zaradi česar tak jezik ne potrebuje avtomatičnega čistilca pomnilnika~\cite{wadler1991there, wadler1990linear, marshall2022linearity}. V linearnem sistemu tipov sta obe pravili prepovedani, s čimer je zagotovljeno, da bo vsaka vrednost uporabljena \textit{natanko} enkrat, po svoji edini uporabi pa lahko tako sistem ustrezno sprosti pomnilnik. Težava, ki se pojavi pri takem jeziku, pa je v tem, da je pogosto preveč omejujoč, saj vrednosti spremenljivk ni mogoče podvajati ali zavreči~\cite{wadler1990linear}. Program \ref{pr:neveljavne-funkcije-linearen-jezik} prikazuje dve funkciji, ki se v jeziku z linearnim sistemom tipov ne bi prevedli. Funkcija \texttt{duplicate} namreč argument \var{x} uporabi dvakrat, funkcija \texttt{fst} pa argumenta \var{y} sploh ne uporabi, kar krši pravila linearnosti.

\begin{primer}[ht]
\centering
\begin{code-box}{Haskell}{Programski jezik z linearnim sistemom tipov \xmark}
duplicate x = (x, x)
fst (x, y) = x
\end{code-box}
\caption{Neveljavni funkciji pri programskem jeziku z linearnim sistemom tipov}
\label{pr:neveljavne-funkcije-linearen-jezik}
\end{primer}

% Uvedba nelinearnih vrednosti & GC
Zaradi teh omejitev se v programske jezike poleg linearnih tipov pogosto uvede še nelinearne~\cite{pierce2004advanced, wadler1990linear, marshall2022linearity}. Nelinearne vrednosti so v takih jezikih posebej označene in omogočajo, da je vrednost uporabljena poljubno mnogokrat, tj. nič ali večkrat. Vendar pa uvedba nelinearnih tipov prinaša izzive pri upravljanju s pomnilnikom. Prevajalnik namreč ne more zanesljivo določiti, kdaj na določeno vrednost ne kaže več nobena referenca, kar pomeni, da je za čiščenje nelinearnih vrednosti v jeziku še vedno potrebna implementacija avtomatičnega čistilca pomnilnika~\cite{wadler1990linear}.

% Wadlerjev sistem linearnih tipov
\subsubsection{Girardova linearna logika}

Sam sistem linearnih tipov temelji na Girardovi linearni logiki~\cite{girard1987linear}. Ta vsebuje tako linearne kot tudi nelinearne tipe, prehajanje med njimi pa je omogočeno s pomočjo pravil promocije (angl. promotion) in opustitve (angl. dereliction).

% Kaj je promocija
Promocija je pravilo, ki omogoča deljenje vrednosti, če je zagotovljeno, da je mogoče deliti tudi vse proste spremenljivke, ki se v vrednosti pojavijo. Če bi bila katera izmed prostih spremenljivk linearna, bi z deljenjem nanjo ustvarili več referenc, kar pa krši pravila linearnosti. Promocija torej omogoča, da linearno vrednost pretvorimo v nelinearno in jo kot tako uporabimo večkrat oziroma sploh ne.

% Kaj je derelikcija (opustitev)
Opustitev je, v kontekstu linearnih tipov, operacija, ki omogoča pretvorbo nelinearnega tipa v linearnega~\cite{wadler1991there}. S pravilom zoženja omogočimo, da lahko nelinearne vrednosti uporabimo večkrat, s pravilom oslabitve o\-mo\-go\-či\-mo, da vrednost sploh ni uporabljena, pravilo opustitve pa omogoči, da nelinearno vrednost uporabimo natanko enkrat, tj. linearno. Brez opustitve v jeziku namreč nelinearnih vrednosti ni mogoče uporabljati kot argumente linearnih funkcij. Toda zaradi pravila opustitve ni mogoče zagotoviti, da ima linearen tip le eno referenco, kar pomeni, da tudi pomnilnika za linearne tipe ni mogoče sprostiti takoj po njihovi prvi uporabi.

% Steadfast tipi
\subsubsection{Wadlerjev sistem stanovitnih tipov}
Wadler v svojem delu~\cite{wadler1990linear} predstavi \textit{len} programski jezik z linearnim sistemom tipov. Tipi so razdeljeni na dve družini, med njima pa \emph{ni mogoče implicitno} prehajati z uporabo promocije oziroma opustitve. Linearni tipi v jeziku predstavljajo reference z možnostjo pisanja (angl. write access), medtem ko nelinearni tipi omogočajo le dostop za branje (angl. read-only access). Za sestavljene podatkovne tipe v jeziku velja, da nelinearni tipi ne smejo vsebovati referenc na linearne tipe. Ker lahko nelinearne tipe podvajamo, bi se v tem primeru namreč lahko zgodilo, da bi podvojili tudi referenco na linearen tip, s čimer pa bi prekršili pravila linearnosti v jeziku.

\begin{primer}[ht]
\centering
\begin{code-box}{Haskell}{Wadlerjev len programski jezik}
let! (x) y = u in v
\end{code-box}
\caption{Izraz \texttt{let!} v Wadlerjevem lenem programskem jeziku}
\label{pr:let-klicaj}
\end{primer}

Prehajanje med linearnimi in nelinearnimi tipi je omogočeno le na en način: \emph{eksplicitno} s pomočjo izraza \texttt{let!} (primer \ref{pr:let-klicaj}). Pri tem je mogoče znotraj izraza $u$ vrednost $x$ uporabljati \emph{nelinearno}, a le za branje. Več kot ena referenca na vrednost v pomnilniku je namreč varna, dokler obstaja \emph{v trenutku posodobitve} nanjo samo ena referenca. V izrazu $v$ pa je tip spremenljivke $x$ ponovno linearen, kar pomeni, da je vrednost mogoče neposredno posodabljati ali izbrisati iz pomnilnika.

Toda zaradi lenosti jezika bi lahko spremenljivka $y$ preživela \texttt{let!} izraz, pri tem pa vsebovala kazalec na spremenljivko $x$, ki je linearno uporabljena v telesu izraza $v$. Izraz \texttt{let!} je zato edini izraz, ki se ne izvaja leno, temveč neučakano. Nujno je namreč izračunati \textit{celoten} izraz $u$, preden se začne izvajati izraz $v$, da zagotovimo, da bodo vse reference na $x$ odstranjene, preden se bo začel izračun $v$, ki bo mogoče sprostil vrednost spremenljivke $x$. To imenujemo tudi za \emph{posebej neučakani izračun} (angl. hyperstrict evaluation).

Wadler v svojem delu torej uvede dva povsem ločena ``svetova`` tipov, med katerimi je moč prehajati z uporabo izraza \texttt{let!}. Tak sistem tipov poimenuje tudi za stanoviten (angl. steadfast) sistem linearnih tipov~\cite{wadler1991there}. Pri tem pokaže, da je zaradi uvedbe nelinearnosti še vedno potreben avtomatski čistilec pomnilnika, saj so nelinearne vrednosti lahko poljubno podvojene. Pokaže tudi, da je potrebno zagotoviti, da se vsi nelinearni dostopi do objekta v pomnilniku izvedejo pred dostopom za pisanje, kar pa je pri lenem izračunu skoraj nemogoče izvesti, zato v \texttt{let!} izraze ponovno uvede neučakan izračun.

Linearni tipi so bili tudi že dodani v Haskell kot razširitev sistema tipov~\cite{bernardy2018linear}. Najpomembnejša pridobitev članka je vpeljava linearnih tipov v Haskell, ki omogoča varno in učinkovito posodabljanje podatkovnih struktur ter zagotavljanje pravilnega dostopa do zunanjih virov, kot so datoteke in omrežni viri. Avtorji so dokazali, da je mogoče linearne tipe vključiti v obstoječi programski jezik s spreminjanjem algoritma za preverjanje in izpeljavo tipov, ne pa tudi s samim spreminjanjem abstraktnega STG stroja, na katerem se izvaja redukcija grafa.

\section{Unikatni tipi}
\label{sec:unikatni-tipi}

% Definicija unikatnih tipov
Unikatni tipi (angl. uniqueness types) so namenjeni zagotavljanju zahteve, da na vsako vrednost kaže natanko ena referenca, kar omogoča učinkovito implementacijo sistema, ki omogoča posodobitve na mestu (angl. in-place updates)~\cite{marshall2022linearity}. V literaturi se unikatni tipi pogosto kar enačijo z linearnimi tipi oziroma se obravnavajo kot posebna vrsta linearnih tipov~\cite{pierce2004advanced, bernardy2018linear}. Za vrednosti linearnih tipov velja, da v \textit{prihodnosti} zagotovo ne bodo podvojene ali zavržene, medtem ko je za unikatne vrednosti zagotovljeno, da v \textit{preteklosti} še niso bile podvojene~\cite{marshall2022linearity, marshall2024functional}.

V praksi pa je pogoj unikatnosti pogosto preveč omejujoč. Včasih namreč želimo, da lahko na isto vrednost kaže več kot en kazalec. V tem primeru se tudi pri unikatnih tipih (podobno kot pri linearnih) uvede \emph{neomejene} (angl. unrestricted) vrednosti. Za vrednosti unikatnega tipa potem velja pogoj unikatnosti referenc, medtem ko lahko na vrednosti z neomejenim tipom kaže poljubno mnogo kazalcev.

Razlika med linearnimi in unikatnimi tipi je v zmožnosti prehajanja med neomejenimi in omejenimi vrednostmi. Kot smo videli v poglavju \ref{sec:linearni-tipi}, lahko pri linearnem sistemu tipov med linearnimi in nelinearnimi vrednostmi prehajamo s pomočjo pravil promocije in opustitve. S pomočjo opustitve je lahko nelinearna spremenljivka v nadaljevanju uporabljena linearno, pri tem pa ni mogoče zagotoviti, da na to nelinearno spremenljivko kaže natanko ena referenca. To pa tudi pomeni, da v linearnem sistemu ne moremo zagotoviti unikatnosti vrednosti.

Pri unikatnih sistemih tipov ni pravila, ki bi omogočala pretvorbo vrednosti neomejenega tipa nazaj v vrednost unikatnega tipa. Ker na neomejene tipe namreč lahko kaže poljubno mnogo kazalcev, jih ni mogoče obravnavati kot unikatne. Kot bomo lahko videli v poglavju \ref{sec:granule}, programski jezik Granule tako pretvorbo omogoča s ključno besedo \texttt{clone}, ki globoko kopira (angl. deep copy) vrednost v pomnilniku. S tem je sicer zagotovljena unikatnost, a je kopiranje precej neučinkovito, saj je potrebno klonirati celoten podgraf v pomnilniku~\cite{marshall2024functional}. Pri sistemih unikatnih tipov torej velja, da je vrednost unikatnega tipa mogoče pretvoriti v neomejeno vrednost, obratno pa ne. Linearni tipi so tako bolj uporabni pri zagotavljanju pravilne uporabe računalniških sredstev (angl. resource), medtem ko sistemi unikatnih tipov omogočajo ponovno uporabo struktur v pomnilniku in posodabljanje le-teh na mestu~\cite{marshall2022linearity}.

% Povezava Wadlerjevih steadfast tipov in unikatnih tipov
V sistemu tipov, kjer mora biti \textit{vsaka} vrednost linearna, je zagotovljeno tudi, da je vsaka vrednost unikatna~\cite{marshall2022linearity}. Linearni tipi namreč ne dovoljujejo podvajanja, zaradi česar je zagotovljeno, da bo referenca na neko vrednost vedno le ena. Wadlerjev sistem stanovitnih linearnih tipov~\cite{wadler1990linear} omeji pravili promocije in opustitve, s čimer v jezik ponovno uvede pogoj za unikatnost reference. Sistem tipov glede na definicijo bolj ustreza sistemu unikatnih tipov, ki pa takrat še ni bil definiran.

\subsubsection{Programski jezik Clean}

% Programski jezik Clean
Eden izmed programskih jezikov, ki uporabljajo sistem unikatnih tipov, je len funkcijski jezik Clean~\cite{smetsers1994guaranteeing}. Za razliko od Haskella, ki za mutacije notranjega stanja in vhodno-izhodne operacije uporablja monade, Clean le-te implementira s pomočjo sistema unikatnih tipov. Prav tako zna prevajalnik unikatne vrednosti spreminjati na mestu, kar zmanjša porabo pomnilnika in omogoča hitrejše izvajanje programov.

Spodnji primer prikazuje program v jeziku Clean. Konstruktor tipa $*T$ predstavlja unikaten tip $T$. Če predpostavljamo, da je \texttt{eat} tipa \texttt{Cake -> Happy} in \texttt{have} tipa \texttt{Cake -> Cake}, potem je program \ref{pr:clean-unikatni-tipi} veljaven. 

\begin{primer}[ht]
\centering
\begin{code-box}{haskell}{Clean \cmark}
possible :: *Cake -> (Happy, Cake)
possible cake = (eat cake, have cake)
\end{code-box}
\caption{Veljaven program v jeziku Clean z unikatnimi tipi}
\label{pr:clean-unikatni-tipi}
\end{primer}

Kot lahko vidimo, se argument \texttt{cake} v telesu funkcije pojavi dvakrat. Funkcija vzame unikaten kazalec na vrednost \texttt{Cake}, ker pa jo v telesu dvakrat uporabi, vrnjena vrednost izgubi unikatnost. Vrnjena vrednost je tako neomejenega tipa.

\section{Programski jezik Granule}
\label{sec:granule}

Programski jezik Granule~\cite{orchard2019quantitative} je \textit{neučakan}, močno tipiziran (angl. strongly typed) funkcijski jezik, ki v svojem sistemu tipov združuje linearne, indeksne in stopenjsko modalne (angl. graded modal) tipe. Granule v svojem sistemu tipov uporablja princip podatkov kot virov (angl. data as a resource). Za upravljanje s pomnilnikom je uporabljen avtomatični čistilec. 

S pomočjo linearnih tipov je v jeziku zagotovljen pogled na podatke kot na fizičen vir, ki mora biti uporabljen enkrat, nato pa nikoli več. Neomejena uporaba nekega vira mora biti v jeziku eksplicitno označena z eksponentno stopenjsko modalnostjo $! \, A$ (pri čemer je $A$ linearen tip, $!A$ pa konstruktor neomejenega tipa), ki označuje, da je vrednost lahko deljena poljubno mnogokrat. Jezik poleg neomejene uporabe omogoča še določanje zgornje meje uporabe podatkov s pomočjo omejene linearne logike (angl. bounded linear logic)~\cite{girard1992bounded}. Tako lahko namesto neomejene uporabe $!A$, določimo zgornjo mejo uporabe vrednosti. Tip $!_2 \, A$ npr. označuje vrednost tipa $A$, ki je lahko uporabljena \textit{največ} dvakrat.

Preverjanje tipov je v jeziku Granule implementirano v dveh stopnjah: najprej se za izraze v programu izpelje trditve in omejitve glede njihovih tipov~\cite{orchard2019quantitative}, nato pa se trditve dokaže s pomočjo dokazovalnika Z3~\cite{demoura2008z3}. Izpeljava tipov (angl. type inference) za globalne (angl. top-level) funkcije ni podprta, zato morajo biti označeni tipi vseh funkcij na globalnem nivoju. V Granule so vse funkcije privzeto linearne, zato se namesto operatorja za linearne funkcije $a \multimap b$, uporablja kar zapis $a \to b$. Neomejeni tipi, kot jih poznamo iz linearne logike, so označeni s pripono \texttt{[]}. Taka oznaka je ekvivalentni oznaki \texttt{!}, ki jo je definiral Girard~\cite{girard1987linear}, omogoča pa poljubno mnogo uporab spremenljivke, tako da omogoči pravili oslabitve in zoženja. Jezik prav tako omogoča omejevanje števila uporab neke spremenljivke s pomočjo pripone \texttt{[n]}, ki določa, da je lahko število uporab spremenljivke največ $n$.

Primer \ref{pr:granule-identiteta} prikazuje identiteto, implementirano v jeziku Granule. Iz oznake tipa (angl. type annotation) lahko prevajalnik razbere, da je \var{id} funkcija, ki sprejme spremenljivko poljubnega tipa in jo zaradi linearnosti (konstruktor tipa $t \to t$) uporabi natanko enkrat.

\begin{primer}[ht]
\centering
\begin{code-box}{text}{Granule \cmark}
id : ∀ {t : Type} . t → t
id x = x
\end{code-box}
\caption{Implementacija identitete v programskem jeziku Granule}
\label{pr:granule-identiteta}
\end{primer}

V primeru \ref{pr:granule-nelinearne-funkcije} sta implementirani funkciji \var{drop} in \var{copy}, ki v jeziku z le linearnimi tipi nista mogoči. Pri obeh funkcijah je označeno število uporab argumenta. Funkcija \var{drop} svojega argumenta ne uporabi, zato je označena s števnostjo \texttt{t [0]}, funkcija \var{copy} pa svoj argument uporabi dvakrat, kar je označeno s števnostjo \texttt{t [2]}.

\begin{primer}[ht]
\centering
\begin{code-box}{text}{Granule \cmark}
drop : ∀ {t : Type} . t [0] → ()
drop [x] = ()

copy : ∀ {t : Type} . t [2] → (t, t)
copy [x] = (x, x)
\end{code-box}
\caption{Nelinearni funkciji \var{drop} in \var{copy}, implementirani v programskem jeziku Granule}
\label{pr:granule-nelinearne-funkcije}
\end{primer}


\subsubsection{Unikatni tipi v jeziku Granule}
% Razširitev Granule: unikatni tipi
V programski jezik Granule je bila poleg linearnih tipov eksperimentalno dodana tudi podpora za unikatne tipe~\cite{marshall2022linearity}. Linearnost v takem jeziku omogoča boljši nadzor nad upravljanjem s sredstvi, medtem ko je unikatnost uporabljena za varno posodabljanje podatkov na mestu. Prevajalnik jezika Granule zna s pomočjo izpeljanih unikatnih tipov generirati optimizirano Haskell kodo, ki lahko spreminja obstoječe sezname in tako ne ustvarja novih kopij. Avtorji so pokazali, da je uporaba unikatnih tipov za delo s tabelami učinkovitejša od uporabe nespremenljivih tabel (angl. immutable arrays). Rezultati so pokazali, da je različica z unikatnimi tabelami hitrejša in porabi bistveno manj časa za upravljanje s pomnilnikom. To je posledica dejstva, da se unikatnim podatkom dodeli prostor v pomnilniku izven kopice GHC prevajalnika in se lahko eksplicitno sprostijo po njihovi uporabi. Avtorji tudi poudarijo, da sistem unikatnih tipov omogoča varno in učinkovito mutacijo podatkov neposredno v funkcijskem jeziku, brez potrebe po uporabi nepreverjene kode (angl. unsafe code), ki je npr. prisotna v Haskell knjižnicah za učinkovite operacije nad tabelami.

\subsubsection{Model lastništva v jeziku Granule}
% Razširitev Granule: ownership model

Pozneje je bil sistem tipov jezika Granule še dodatno razširjen s pravili za lastništvo in izposojo na podlagi tistih iz Rusta~\cite{marshall2024functional}. Avtorji v članku povežejo koncepta linearnih in unikatnih tipov in vgradijo sistem lastništva in izposoje v sistem tipov za funkcijski programski jezik.

Sistem lastništva v Rustu določa, da z vsako vrednostjo v pomnilniku upravlja natanko ena referenca. To se ujema z definicijo unikatnih vrednosti, zato avtorji osnovo za lastništvo objektov postavijo na sistem unikatnih tipov. Ker vrednosti unikatnega tipa ni mogoče podvajati, avtorji uvedejo ključno besedo \texttt{clone}, ki omogoča globoko kloniranje (angl. deep copy) objekta v pomnilniku. Po operaciji na novonastali objekt zagotovo kaže le en kazalec, s čimer je omogočeno, da lahko vrednost poljubnega tipa pretvorimo v unikaten tip. Dodana je še ključna beseda \texttt{share}, ki omogoča deljenje izraza. Pri deljenju se vrednost unikatnega tipa $*A$ pretvori v neomejeno vrednost $!A$, ki je od takrat naprej ni mogoče ponovno pretvoriti v unikaten tip. Za čiščenje pomnilnika neomejenih tipov se še vedno uporablja avtomatski čistilec, medtem ko se za čiščenje unikatnih in linearnih tipov uporablja Rustov model upravljanja s pomnilnikom.

Izposoje so v jeziku implementirane s pomočjo delnih pravic (angl. fractional permissions)~\cite{boyland2003checking}. Tipi z delnimi pravicami so označeni $\&_p A$, kjer $p$ predstavlja bodisi vrednost $*$ bodisi ulomek na intervalu $[0, 1]$. Vrednosti tipa $\&_{*} A$ predstavljajo unikatne izposoje in so z vrednostmi unikatnih tipov povezane s pomočjo enakosti $*A \equiv \&_{*} A$. Spremenljive izposoje so označene s tipom $\&_1 A$, pri nespremenljivih izposojah pa je $p < 1$. Z izrazom \texttt{split} je omogočeno, da se referenca $\&_p A$ razdeli na dve novi referenci, ki kažeta na isti objekt kot prvotna referenca. Novi referenci sta označeni s polovico dovoljenj prvotne reference, tj. referenci imata tipa $\&_{\frac{p}{2}} A$. Z izrazom \texttt{join} se dve obstoječi referenci združita v eno novo referenco, pri čemer se dovoljenja združenih referenc seštejeta~\cite{marshall2024functional}. Tako lahko spremenljivo referenco s pomočjo izraza \texttt{split} razdelimo na dve nespremenljivi referenci in s pomočjo izraza \texttt{join} ponovno združimo v spremenljivo referenco.

\chapter{Implementacija}
\label{ch:implementacija}

% Naša prevajalnik temelji na 
% Naš prevajalnik:
%   * razčlenjevanje (angl. parsing)
%   * semantična analiza

V fazi semantične analize prevajalnik izvede najprej razreševanje imen (angl. name resolution), ki mu sledita analiza izposoj (angl. borrow check) in premikov (angl. move check).

Pri razreševanju imen prevajalnik preveri, ali so vsa imena spremenljivk definirana pred njihovo uporabo. Prevajalnik se v tej fazi rekurzivno sprehodi čez abstraktno sintaksno drevo STG jezika. Pri tem vzdržuje kontekst trenutno živih spremenljivk. Pri uporabi spremenljivk prevajalnik preveri, ali se ime nahaja v kontekstu in v nasprotnem primeru vrne napako. Če se proces razreševanja imen zaključi brez napake, je zagotovljeno, da tekom izvajanja programa ne bo prišlo do napake zaradi uporabe nedefinirane spremenljivke.

Pri analizi izposoj prevajalnik zagotovi, da izposoja ne živi dlje od spremenljivke, ki jo referencira. 

\begin{stgcode}
-- Glavna funkcija
main = THUNK(
	let a = THUNK(12) in
		&a
)
\end{stgcode}

\begin{figure}[ht]
	\centering
	\begin{tikzpicture}
	\tikzset{
		every node/.append style={text width=1.4cm,execute at begin node=\setlength{\baselineskip}{1em},font=\footnotesize},
		block/.style={draw,rectangle,text width=2cm,align=center,minimum height=1cm,minimum width=2cm},
	}
	
	\node[block] (parser) {\textbf{Razčlen\-jevanje}};
	
	\node[coordinate, left=0.5cm of parser.west] (levo-od-parser) {};
	
	\node[block,right=1cm of parser] (borrow-checker) {Analiza izposoj};
	\node[block, above=0.5cm of borrow-checker] (name-resolution) {\textbf{Raz\-re\-še\-van\-je imen}};
	\node[block, below=0.5cm of borrow-checker] (move-checker) {Analiza premikov};
	\node[block, right=1cm of borrow-checker] (interpreter) {\textbf{Abstraktni stroj STG'}};
	
	\node[coordinate, right=0.5cm of interpreter.east] (desno-od-interpreter) {};
	
	% Input arrow
	\draw[->] (levo-od-parser) -- (parser) node[above, pos=-0.25,align=center] {\scriptsize tok\\znakov};
	
	% Output arrow
	\draw[->] (interpreter) -- (desno-od-interpreter) node[above,pos=1.25,align=center] {\scriptsize rezultat};
	
	% Parser -> name resolution arrow
	\node[coordinate, right=0.3cm of parser.east] (desno-od-parser) {};
	\node[coordinate] at (desno-od-parser |- name-resolution) (levo-od-name-resolution) {};	
	\draw[->] (parser) -- (desno-od-parser) -- node[align=center,sloped,anchor=center,above,pos=0.75] {\scriptsize STG} (levo-od-name-resolution) -- (name-resolution);
	
	% Move checker -> abstract STG machine arrow
	\node[coordinate, left=0.3 of interpreter.west] (levo-od-interpreter) {};
	\node[coordinate] at (move-checker -| levo-od-interpreter) (desno-od-move-checker) {};
	\draw[->] (move-checker.east) -- (desno-od-move-checker) -- node[align=center,sloped,anchor=center,below,pos=0.25] {\scriptsize STG'} (levo-od-interpreter) -- (interpreter);
	
	% 
	\draw[->] (name-resolution) -- node[pos=0.5,right] {\scriptsize STG} (borrow-checker);
	\draw[->] (borrow-checker) -- node[pos=0.5,right] {\scriptsize STG'} (move-checker);
	
	% Oznake za semantično analizo
	\node[coordinate, above=0.4 of name-resolution.north] (top) {};
	\node[coordinate, below=0.4 of move-checker.south] (bottom) {};
	
	\node[coordinate, left=0.5 of borrow-checker.west] (semanticna-analiza-levo) {};
	\node[coordinate, right=0.5 of borrow-checker.east] (semanticna-analiza-desno) {};
	
	\node[coordinate] at (semanticna-analiza-levo |- top) (semanticna-analiza-levo-zgoraj) {};
	\node[coordinate] at (semanticna-analiza-levo |- bottom) (semanticna-analiza-levo-spodaj) {};
	
	\node[coordinate] at (semanticna-analiza-desno |- top) (semanticna-analiza-desno-zgoraj) {};
	\node[coordinate] at (semanticna-analiza-desno |- bottom) (semanticna-analiza-desno-spodaj) {};
	
	\draw[-,dashed] (semanticna-analiza-levo-zgoraj) -- (semanticna-analiza-levo-spodaj) -- (semanticna-analiza-desno-spodaj) -- (semanticna-analiza-desno-zgoraj) -- (semanticna-analiza-levo-zgoraj);
	
	\path (semanticna-analiza-levo-zgoraj) -- node[above=0.1,align=center,text width=3cm] {Semantična analiza} (semanticna-analiza-desno-zgoraj);
	
	\end{tikzpicture}
	\caption{Faze implementirane prevajalnika}
	\label{fig:shema-implementacije}
\end{figure}

\section{Analiza izposoj}

\section{Analiza premikov}

\chapter{Rezultati}
\label{ch:rezultati}

% Za potrebe naše magistrske naloge bomo v izbranem programskem jeziku implementirali simulator STG stroja. V programskem jeziku STG bomo napisali zbirko programov, s pomočjo katerih bomo testirali uspešnost implementirane metode. Merili bomo količino dodeljenega pomnilnika in količino sproščenega pomnilnika in skušali ugotoviti, ali je ves pomnilnik pravočasno sproščen. Cilj magistrskega dela ni izdelava učinkovite implementacije čiščenja pomnilnika, temveč skušati ugotoviti, kakšne spremembe in analize je potrebno dodati v STG stroj, da bo lahko uporabljal princip lastništva namesto čistilca pomnilnika.

%\chapter{Zaključek}
%\label{ch:zakljucek}

%----------------------------------------------------------------
% SLO: bibliografija
% ENG: bibliography
%----------------------------------------------------------------
\bibliographystyle{elsarticle-num}
\bibliography{bibliography}

\end{document}
