\chapter{Implementacija lenega izračuna z uporabo STG}
\label{ch:stg}

% Uvod: Imperativni vs funkcijski programski jeziki
Glede na paradigmo delimo programske jezike na imperativne in deklarativne. Imperativni jeziki, kot so npr. C, Rust, Python, \dots, iz vhodnih podatkov izračunajo rezultat s pomočjo stavkov, ki spreminjajo notranje stanje programa oziroma stroja na katerem se izvajajo. Ukazi zelo podrobno opisujejo samo izvajanje programa, in jih je zato nekoliko lažje prevesti v strojne ukaze, ki jih zna izvajati procesor. Pri deklarativnem programiranju pa je program podan kot višjenivojski opis želenega rezultata, ki ne opisuje dejanskega poteka programa. Med deklarativne jezike štejemo npr. SQL, katerega program je opis podatkov, ki jih želimo pridobiti in kako jih želimo manipulirati, ne da bi podrobno opisovali kako naj sistem izvede te operacije.

% Funkcijski programski jeziki
Med deklarativne programske jezike pa spadajo tudi funkcijski, pri katerih je glavna operacija nad podatki ravno funkcija. Funkcijski programi so sestavljeni iz zaporedja aplikacij (angl. application) in kompozicij funkcij. V modernejših funkcijskih programskih jezikih, so funkcije višjenivojske (angl. higher order)~\cite{hudak1989conception}, kar pomeni, da jih je mogoče prirediti v spremenljivke, jih uporabiti kot argumente in kot rezultate drugih funkcij, ter da omogočajo rekurzijo in polimorfizem. Prav tako taki jeziki pogosto omogočajo in celo spodbujajo tvorjenje novih funkcij z uporabo curryjinga oziroma delne aplikacije~\cite{hudak1989conception}, pri kateri je funkciji podanih le del njenih argumentov.

Čisti funkcijski programski jeziki spodbujajo uporabo čistih funkcij (angl. pure functions)~\cite{hudak1989conception}. Te za izračun vrednosti rezultata uporabljajo le vrednosti svojih argumentov, ne pa tudi drugega globalnega stanja in ne povzročajo stranskih učinkov (angl. side effects). Med take jezike spada npr. Haskell, pri katerem program opišemo kot kompozicijo čistih funkcij, ki vhod preslikajo v izhod. Vhodno-izhodne operacije so potisnjene na rob samega računanja, saj povzročajo stranske učinke, ki pa so v jeziku neželene. Stranski učinki so v jeziku predstavljeni eksplicitno z uporabo monad. 

V nadaljevanju se bomo podrobneje posvetili funkcijskemu programskemu jeziku Haskell, ki temelji na čistih funkcijah in lenosti. Osredotočili se bomo na delovanje prevajalnika GHC (angl. Glasgow Haskell compiler), ki izvorno kodo prevede v strojno kodo. Pri prevajanju se program transformira v več različnih vmesnih predstavitev (angl. intermediate representation), mi pa se bomo v magistrskem delu osredotočili predvsem na vmesno kodo imenovano STG jezik (angl. Spineless tagless G-Machine language), katerega delovanje bomo podrobneje opisali v razdelku \ref{sec:stg-definicija}.

\section{Prevajalnik GHC}
\label{sec:prevajalnik-ghc}

Prevajalnik GHC prevajanje iz izvorne kode v programskem jeziku Haskell v strojno kodo izvaja v več zaporednih fazah oziroma modulih~\cite{jones1993glasgow, brown2012architecture}. Vsaka faza kot vhod prejme izhod prejšnje, nad njim izvede določeno transformacijo in rezultat posreduje naslednji fazi. Faze glede na njihovo funkcijo v grobem delimo na tri dele. V prednjem delu (angl. front-end) se nad izvorno kodo najprej izvede leksikalna analiza, pri kateri se iz toka znakov, ki predstavljajo vhodni program pridobi abstraktno sintaksno drevo (angl. abstract syntax tree). Nad drevesom se izvede še zaporedje semantičnih analiz pri katerih se preveri ali je program pomensko pravilen. Sem sodi razreševanje imen, pri kateri se razreši vsa imena spremenljivk iz uvoženih modulov v programu in preveri ali so vse spremenljivke deklarirane pred njihovo uporabo. Izvede se še preverjanje tipov, kjer se za vsak izraz izpelje njegov najbolj splošen tip in preveri ali se vsi tipi v programu ujemajo.

\begin{figure}[h]
	\centering
	\begin{tikzpicture}
		\tikzset{
			every node/.append style={text width=1.4cm,execute at begin node=\setlength{\baselineskip}{1em},font=\footnotesize},
			block/.style={draw,rectangle,text width=2cm,align=center,minimum height=1cm,minimum width=2cm},
		}
		
		% Srednji del prevajalnika
		\node[block] (optimizacija) {Optimizacija};
		\node[coordinate, left=2.6cm of optimizacija] (center-levo) {};
		\node[coordinate, right=2.6cm of optimizacija] (center-desno) {};
		
		% Prednji del prevajalnika
		\node[block, above=0.55cm of center-levo] (semanticna-analiza) {Semantična analiza};
		\node[block, above=0.55cm of semanticna-analiza] (razclenjevanje) {Razčlen\-je\-van\-je};
		\node[block, below=0.55cm of semanticna-analiza] (izpeljava-tipov) {Izpeljava tipov};
		\node[block, below=0.55cm of izpeljava-tipov] (razsladenje) {Razsladenje};
		
		% Zadnji del prevajalnika
		\node[block, above=0.55cm of center-desno] (stg-to-cmm) {Izbira \texttt{C--} ukazov};
		\node[block, above=0.55cm of stg-to-cmm] (core-to-stg) {Izbira STG ukazov};
		
		% Moznosti prevajanja C-- -> strojna koda
		\node[block, minimum height=0.6cm, below=0.9cm of stg-to-cmm] (neposredno) {Neposredno};
		\node[block, minimum height=0.6cm, below=0.3cm of neposredno] (gcc) {GCC};
		\node[block, minimum height=0.6cm, below=0.3cm of gcc] (llvm) {LLVM};
		
		% 
		\node[coordinate, left=0.4cm of neposredno.west] (levo-od-neposredno) {};
		\node[coordinate, left=0.4cm of gcc.west] (levo-od-gcc) {};
		\node[coordinate, left=0.4cm of llvm.west] (levo-od-llvm) {};
		\draw[-] (levo-od-neposredno) -- (levo-od-gcc) -- (levo-od-llvm);
		\draw[->] (levo-od-neposredno) -- (neposredno);
		\draw[->] (levo-od-gcc) -- (gcc);
		\draw[->] (levo-od-llvm) -- (llvm);
		
		\node[coordinate, right=0.4cm of neposredno.east] (desno-od-neposredno) {};
		\node[coordinate, right=0.4cm of gcc.east] (desno-od-gcc) {};
		\node[coordinate, right=0.4cm of llvm.east] (desno-od-llvm) {};
		\draw[-] (desno-od-neposredno) -- (desno-od-gcc) -- (desno-od-llvm);
		\draw[-] (neposredno) -- (desno-od-neposredno);
		\draw[-] (gcc) -- (desno-od-gcc);
		\draw[-] (llvm) -- (desno-od-llvm);
		
		\node[coordinate, below=0.55cm of stg-to-cmm] (pod-stg-to-cmm) {};
		\node[coordinate] at (pod-stg-to-cmm -| levo-od-neposredno) (nad-levo-od-neposredno) {};
		\draw[-] (stg-to-cmm) -- (pod-stg-to-cmm) node[pos=0.5,right] {\scriptsize \texttt{C--}} -- (nad-levo-od-neposredno) -- (levo-od-neposredno) ;
		
		\draw[->] (razclenjevanje) -- (semanticna-analiza) node[pos=0.5,right] {\scriptsize AST};
		\draw[->] (semanticna-analiza) -- (izpeljava-tipov) node[pos=0.5,right] {\scriptsize AST};
		\draw[->] (izpeljava-tipov) -- (razsladenje) node[pos=0.5,right] {\scriptsize AST};
		
		\draw[->] (core-to-stg) -- (stg-to-cmm) node[pos=0.5,right] {\scriptsize STG};
		
		% Puscica "Generator vmesne kode" -> "Optimizacija vmesne kode"
		\node[coordinate, left=0.5cm of optimizacija.west] (levo-od-optimizacija) {};
		\node[coordinate] at (razsladenje -| levo-od-optimizacija) (desno-od-razsladenje) {};
		
		\draw[->] (razsladenje.east) -- (desno-od-razsladenje) -- (levo-od-optimizacija) node[pos=0.5,right] {\scriptsize Core} -- (optimizacija);
		
		% Puscica "Optimizacija vmesne kode" -> "Izbira ukazov"
		\node[coordinate, right=0.5cm of optimizacija.east] (desno-od-optimizacija) {};
		\node[coordinate] at (core-to-stg -| desno-od-optimizacija) (levo-od-izbira-ukazov) {};
		
		\draw[->] (optimizacija) -- (desno-od-optimizacija) -- (levo-od-izbira-ukazov) node[pos=1.0,left,align=right] {\scriptsize Core} -- (core-to-stg);
		
		% Vhodna povezava "Zacetek" -> "Leksikalna analiza"
		\node[coordinate, left=2cm of center-levo] (zacetek) {};
		\node[coordinate, left=0.5cm of razclenjevanje] (levo-od-razclenjevanje) {};
		\node[coordinate] at (zacetek -| levo-od-razclenjevanje) (desno-od-zacetek) {};
		
		\draw[->] (zacetek) -- (desno-od-zacetek) -- (levo-od-razclenjevanje) node[pos=0.2,left,text width=0.9cm] {\scriptsize tok\\znakov} -- (razclenjevanje);
		
		% Koncna povezava
		\node[coordinate, right=2cm of center-desno] (konec) {};
		\node[coordinate] at (konec -| desno-od-neposredno) (levo-od-konec) {};
		\draw[->] (desno-od-neposredno) -- (levo-od-konec) -- (konec) node[pos=1,above] {\scriptsize strojna\\koda};
		
		% Locevalne crte
		\node[coordinate, left=0.25cm of levo-od-optimizacija] (sprednji-del-center) {};
		\node[coordinate, above=3.7cm of sprednji-del-center] (sprednji-del-zgoraj) {};
		\node[coordinate, below=3.2cm of sprednji-del-center] (sprednji-del-spodaj) {};
		\draw[dashed] (sprednji-del-zgoraj) -- (sprednji-del-spodaj) node[pos=0,left=0.2cm,text width=2cm,align=right] {prednji del} node[pos=0,right=0.2cm,text width=2cm] {srednji del};
		
		\node[coordinate, right=0.25cm of desno-od-optimizacija] (zadnji-del-center) {};
		\node[coordinate, above=3.7cm of zadnji-del-center] (zadnji-del-zgoraj) {};
		\node[coordinate, below=3.2cm of zadnji-del-center] (zadnji-del-spodaj) {};
		\draw[dashed] (zadnji-del-zgoraj) -- (zadnji-del-spodaj) node[pos=0,right=0.2cm,text width=2cm] {zadnji del};
		
	\end{tikzpicture}
	\caption{Pomembnejše faze prevajalnika Glasgow Haskell compiler}
	\label{fig:shema-ghc}
\end{figure}

Bogata sintaksa programskega jezika Haskell predstavlja velik izziv za izdelavo prevajalnikov, saj zahteva natančno prevajanje raznolikih sintaktičnih struktur in konstruktov v strojno kodo. Težavo rešuje zadnji korak prednjega dela prevajalnika, imenovan razsladenje (angl. desugarification). V njem se sintaktično drevo jezika Haskell pretvori v drevo jezika Core, ki je minimalističen funkcijski jezik, osnovan na lambda računu. Kljub omejenemu naboru konstruktov omogoča Core zapis katerega koli Haskell programa. Vse nadaljnje faze prevajanja se tako izvajajo na tem precej manjšem jeziku, kar močno poenostavi celoten proces.

Srednji del (angl. middle-end) prevajalnika sestavlja zaporedje optimizacij, ki kot vhod sprejmejo program v Core jeziku in vrnejo izboljšan program v Core jeziku. Rezultat niza optimizacij se posreduje zadnjemu delu (angl. back-end) prevajalnika, ki poskrbi za prevajanje Core jezika v strojno kodo, ki se lahko neposredno izvaja na procesorju. Na tem mestu se Core jezik prevede v STG jezik, ta pa se nato prevede v programski jezik \texttt{C--}. Slednji je podmnožica programskega jezika C in ga je mogoče v strojno kodo prevesti na tri načine: neposredno ali z enim izmed prevajalnikov LLVM ali GCC. Prednost take vrste prevajanja je v večji prenosljivosti programov, saj znata LLVM in GCC generirati kodo za večino obstoječih procesorskih arhitektur, poleg tega pa imata vgrajene še optimizacije, ki pohitrijo delovanje izhodnega programa.

\section{Definicija STG jezika}
\label{sec:stg-definicija}

Kot smo si podrobneje pogledali v poglavju \ref{sec:leni-izracun}, lene funkcijske programske jezike najpogosteje implementiramo s pomočjo redukcije grafa. Eden izmed načinov za izvajanje redukcije je abstraktni STG stroj (angl. Spineless Tagless G-machine)~\cite{jones1992implementing, marlow2004making}, ki definira in zna izvajati majhen funkcijski programski jezik STG. STG stroj in jezik se uporabljata kot vmesni korak pri prevajanju najpopularnejšega lenega jezika Haskell v prevajalniku GHC (Glasgow Haskell Compiler)~\cite{GHC}. Sledi formalna definicija STG jezika. Pri tem bomo spremenljivke oz\-na\-če\-va\-li s poševnimi malimi tiskanimi črkami $x, y, f, g$, konstruktorje pa s poševnimi velikimi tiskanimi črkami $C$.

\begin{align*}
	\text{konstanta} \quad \coloneq& \quad \underline{int} \enspace \vert \enspace \underline{double} & \text{primitivne vrednosti}
\end{align*}

STG jezik podpira dva neokvirjena (angl. unboxed) podatkovna tipa: celoštevilske vrednosti in števila s plavajočo vejico. Neokvirjene primitivne tipe označujemo z \texttt{n\#}. Poleg tega jezik omogoča uvajanje novih algebraičnih podatkovnih tipov, ki jih lahko tvorimo oziroma inicializiramo s pomočjo konstruktorjev $C$. Pri tem je vredno omeniti, da so algebraični podatkovni tipi definirani v jeziku, ki ga prevajamo v STG, v našem primeru torej v Haskellu. Prav tako se v Haskellu izvaja tudi izpeljava in preverjanje tipov, abstraktni STG stroj pa med samim prevajanjem nima informacij o tipih.

\begin{align*}
	a, v \quad \coloneq& \quad \text{konstanta} \enspace \vert \enspace x & \text{argumenti so atomarni}
\end{align*}

Vsi argumenti pri aplikaciji funkcij in primitivnih operacij so v A-normalni obliki (angl. A-normal form)~\cite{flanagan1993essence}, kar pomeni, da so atomarni (angl. atomic). Tako je vsak argument ali primitivni podatkovni tip ali pa spremenljivka. Pri prevajanju v STG jezik lahko prevajalnik sestavljene argumente funkcij priredi novim spremenljivkam z ovijanjem v \texttt{let} izraz in spremenljivke uporabi kot argumente pri klicu funkcije. Pri tem je potrebno zagotoviti, da so definirane spremenljivke unikatne oziroma, da se ne pojavijo v ovitem izrazu. Aplikacijo funkcije $f \; (\oplus \; x \; y)$ bi tako ovili v  izraz $\texttt{let} \enspace a = \oplus \; x \; y \enspace \texttt{in} \enspace f \enspace a$, s čemer bi zagotovili, da so vsi argumenti atomarni.

\begin{align*}
	k \quad \coloneq& \quad \bullet & \text{neznana mestnost funkcije}\\
	\vert& \quad n & \text{znana mestnost $n \geq 1$}\\
\end{align*}

Prevajalnik lahko med prevajanjem za določene funkcije določi njihovo mestnost (angl. arity), tj. število argumentov, ki jih funkcija sprejme. Ker pa je STG funkcijski jezik, lahko funkcije nastopajo tudi kot argumenti drugih funkcij, zato včasih določevanje mestnosti ni mogoče. V teh primerih funkcije označimo z neznano mestnostjo in jim med izvajanjem posvetimo posebno pozornost. Povsem veljavno bi bilo vse funkcije v programu označiti z neznano mestnostjo $\bullet$, a je mogoče s podatkom o mestnosti klice funkcij implementirati bolj učinkovito, zato se med prevajanjem izvaja tudi analiza mestnosti.

\begin{align*}
	expr \quad \coloneq& \quad a & \text{atom}\\
	\vert& \quad f^k \: a_1 \dots a_n & \text{aplikacija funkcije ($n \geq 1$)}\\
	\vert& \quad \oplus \: a_1 \dots a_n & \text{primitivna operacija ($n \geq 1$)}\\
	\vert& \quad \texttt{let} \enspace x = obj \enspace \texttt{in} \enspace e & \text{} \\
	\vert& \quad \texttt{case} \enspace e \enspace \texttt{of} \enspace \{ alt_1; \dots; alt_n \}& \text{} \\
\end{align*}

Primitivne operacije so funkcije implementirane v izvajalnem okolju (angl. runtime) in so namenjene izvajanju računskih operacij nad neokvirjenimi primitivnimi podatki. Jezik podpira celoštevilske operacije \texttt{+\#}, \texttt{-\#}, \texttt{*\#}, \texttt{/\#}, in operacije nad števili s plavajočo vejico. Pri tem velja, da so vse primitivne operacije \textit{zasičene}, kar pomeni, da sprejmejo natanko toliko argumentov, kot je mestnost (angl. arity) funkcije. Če programski jezik omogoča delno aplikacijo primitivnih funkcij, potem je potrebno take delne aplikacije z $\eta$-dopolnjevanjem razširiti v nasičeno obliko. Pri tem delno aplikacijo ovijemo v nove lambda izraze z uvedbo novih spremenljivk, ki se ne pojavijo nikjer v izrazu. Tako npr. izraz \texttt{(+ 3)}, ki predstavlja delno aplikacijo vgrajene funkcije za seštevanje prevedemo v funkcijo $\lambda x . (+ \: 3 \: x)$ in s tem zadostimo pogoju zasičenosti.

Izraz \texttt{let} na kopici ustvari nov objekt in je kot tak tudi edini izraz, ki omogoča alokacijo novih objektov na pomnilniku. Objekti na kopici bodo podrobneje opisani v nadaljevanju. V naši poenostavljeni različici STG jezika, izraz \texttt{let} ne omogoča rekurzivnih definicij. Edini vir rekurzije je v jeziku omogočen s pomočjo statičnih definicij (angl. top-level definitions), ki se lahko rekurzivno sklicujejo med sabo oziroma same nase.

\begin{align*}
	alt \quad \coloneq& \quad C \enspace x_1 \dots x_n \to expr & \text{algebraična alternativa}\\
	\vert& \quad x \to expr & \text{privzeta alternativa}\\
\end{align*}

Izraz \texttt{case} je sestavljen iz podizraza $e$, ki se izračuna (angl. scrutinee) in seznama alternativ $alts$, od katerih se vedno izvede \textit{natanko ena}. S preverjanjem tipov v fazi pred prevajanjem v STG je zagotovljeno, da vsi konstruktorji pri alternativah spadajo pod isti algebraični vsotni podatkovni tip (angl. sum type) in da so alternative \textit{izčrpne}, tj. da obstaja privzeta alternativa ali da algebraične alternative pokrijejo ravno vse možne konstruktorje podatkovnega tipa. Izraz \texttt{case} najprej shrani vse žive spremenljivke, ki so uporabljene v izrazih v alternativah, na sklad doda kontinuacijo (angl. continuation), tj. naslov kjer se bo izvajanje nadaljevalo in nato začne računati vrednost izraza $e$. Zaradi bolj učinkovitega izvajanja, se pri prevajanju za konstruktorje vsotnih podatkovnih tipov generirajo oznake, navadno kar nenegativne celoštevilske vrednosti, ki se hranijo v objektu \textsc{con}. Pri razstavljanju sestavljenega podatkovnega tipa v \texttt{case} izrazu lahko tako primerjamo cela števila in ne celih nizov.

\begin{align*}
	obj \quad \coloneq& \quad \textsc{fun}(x_1 \dots x_n \to e) & \text{aplikacija}\\
	\vert& \quad \textsc{pap}(f \; a_1 \dots a_n) & \text{delna aplikacija}\\
	\vert& \quad \textsc{con}(C \; a_1 \dots a_n) & \text{konstruktor}\\
	\vert& \quad \textsc{thunk} \enspace e & \text{zakasnitev}\\
	\vert& \quad \textsc{blackhole} & \text{črna luknja}
\end{align*}

Kot smo že omenili, se novi objekti na kopici tvorijo le s pomočjo \texttt{let} izraza. Jezik STG podpira pet različnih vrst objektov, ki se razlikujejo glede na oznako v ovojnici na pomnilniku.

Objekt \textsc{fun} predstavlja funkcijsko ovojnico (angl. closure) z argumenti $x_1, \dots, x_n$ in telesom $e$, ki pa se lahko poleg argumentov $x_i$ sklicuje še na druge proste spremenljivke. Pri tem velja, da je lahko funkcija aplicirana na več kot $n$ ali manj kot $n$ argumentov, tj. je curryrana.
% Andrej Bauer pravi, da je to okej.
% https://x.com/andrejbauer/status/621602561368399872

Objekt \textsc{pap} predstavlja delno aplikacijo (angl. partial application) funkcije $f$ na argumente $x_1, \dots, x_n$. Pri tem je zagotovljeno, da bo $f$ objekt tipa \textsc{fun}, katerega mestnost bo \textit{vsaj} $n$.

Objekt \textsc{con} predstavlja nasičeno aplikacijo konstruktorja $C$ na argumente $a1, \dots a_n$. Pri tem je število argumentov, ki jih prejme konstruktor natančno enako številu parametrov, ki jih zahteva.

Objekt \textsc{thunk} predstavlja zakasnitev izraza $e$. Kadar se vrednost izraza uporabi, tj. kadar se izvede \texttt{case} izraz, se izračuna vrednost $e$, \textsc{thunk} objekt na kopici pa se nato posodobi s preusmeritvijo (angl. indirection) na vrednost $e$. Pri izračunu zakasnitve se objekt \textsc{thunk} na kopici zamenja z objektom \textsc{blackhole}, s čemer se preprečuje puščanje pomnilnika~\cite{jones1992tail} in neskončnih rekurzivnih struktur. Objekt \textsc{blackhole} se lahko pojavi le kot rezultat evalvacije zakasnitve, nikoli pa v vezavi v \texttt{let} izrazu.

\begin{align*}
	program \quad \coloneq& \quad f_1 = obj_1 \: ; \: \dots \: ; \: f_n = obj_n
\end{align*}

Program v jeziku STG je zaporedje vezav, ki priredijo STG objekte v spremenljivke.

\section{Operacijska semantika}
\label{sec:operacijska-semantika}

Operacijska semantika malih korakov bo opisana s pravili podanimi v obliki, ki jo prikazuje enačba \ref{eq:operacijska-semantika-oblika-pravil}. Pri tem s $P$ označujemo predikate, ki morajo biti izpolnjeni, da se pravilo izvede, z $e$ označujemo izraze (iz poglavja \ref{sec:stg-definicija}), oznaka $s$ predstavlja sklad kontinuacij, $H$ pa kopico.

\begin{equation}
    \infer[korak]{P}{
	e_1; \, s_1; \, H_1
}{
	e_2; \, s_2; \, H_2
}
	%e_1; \, s_1; \, H_1  \;\; \Rightarrow  \;\; e_2; \, s_2; \, H_2
	\label{eq:operacijska-semantika-oblika-pravil}
\end{equation}

Na skladu $s$ hranimo kontinuacije, ki stroju povedo, kaj storiti, ko bo trenutni izraz $e$ izračunan. Zapis $s = k : s'$ označuje sklad $s$, kjer je $k$ trenutni element na vrhu sklada, $s'$ pa predstavlja preostanek sklada pod njim. Kontinuacije so lahko ene izmed naslednjih oblik:

\begin{align*}
	k \quad \coloneq& \quad \texttt{case} \; \bullet \; \texttt{of} \; \{ alt_1; \dots; alt_n \}\\
	\vert& \quad \textit{Upd} \; t \; \bullet\\
	\vert& \quad (\bullet \; a_1 \dots a_n)
\end{align*}

Kontinuacija $\texttt{case} \; \bullet \; \texttt{of} \; \{ alt_1; \dots; alt_n \}$ izvede glede na trenutno vrednost $e$ natanko eno izmed alternativ $alt_1, \dots, alt_n$. Pri kontinuaciji $\textit{Upd} \; t \; \bullet$ se zakasnitev na naslovu $t$ posodobi z vrnjeno vrednostjo, tj. trenutnim izrazom $e$. Zadnja kontinuacija $(\bullet \; a_1 \dots a_n)$ pa uporabi vrnjeno funkcijo nad argumenti $a_1, \dots, a_n$. Sestavimo jo, kadar je pri klicu funkcije argumentov preveč. Če vemo, da funkcija sprejme $n$ argumentov, pri aplikaciji pa je podanih $n + k$ argumentov, bo na sklad dodana kontinuacija $(\bullet \; a_{n + 1 } \dots a_{ n + k })$ s $k$ argumenti.

Kopica $H$ je v operacijski semantiki predstavljena kot slovar (angl. map), ki spremenljivke slika v objekte na kopici. Oznaka $H[x]$ predstavlja dostop naslova $x$ na kopici $H$. Imena spremenljivk se torej v operacijski semantiki kar enačijo s pomnilniškimi naslovi. V dejanski implementaciji je kopica kar zaporeden kos pomnilnika, do katerega dostopamo preko pomnilniških naslovov, naloga prevajalnika pa je, da spremenljivkam dodeli pomnilniške naslove in s tem zagotovi pravilno preslikavo spremenljivk na objekte v pomnilniku. Prav tako je vsak objekt \textsc{fun}, \textsc{thunk}, \dots na kopici predstavljen kot funkcijska ovojnica, ki hrani naslove oziroma vrednosti prostih spremenljivk, ki se v njem pojavijo.

Omenimo še, da je pri pravilih operacijske semantike vrstni red pomemben, saj se pri izvajanju izvede prvo pravilo, ki ustreza trenutnemu stanju abstraktnega STG stroja.

\subsubsection{Dodeljevanje novih objektov na pomnilniku}

Pravilo \ref{eq:stg-let} poda operacijsko semantiko \texttt{let} izraza. Ta na kopici ustvari nov objekt $obj$ in ga priredi novi unikatni spremenljivki $x'$, ki ni uporabljena nikjer v programu, kar ustreza alokaciji še neuporabljenega naslova na kopici. Zapis $e[x' / x]$ predstavlja izraz $e$, v katerem so vse proste spremenljivke $x$ zamenjane z vrednostjo $x'$. Pri \texttt{let} izrazu se torej najprej alocira nov objekt na pomnilniku, nato pa se izvede telo, v katerem so vse pojavitve spremenljivke $x$ zamenjane z novim pomnilniškim naslovom.

\begin{equation}
\infer[let]{
	\text{$x'$ je sveža spremenljivka}
}{
	\texttt{let} \enspace x = obj \enspace \texttt{in} \enspace e; \, s; \, H
}{
	e[x'/x]; \, s; \, H
}
\label{eq:stg-let}
\end{equation}

V pravi implementaciji je substitucija $e[x'/x]$ implementirana kot prepisovanje spremenljivke $x$ s kazalcem na objekt $x'$ v klicnem zapisu funkcije.

\subsubsection{Pogojna izbira s stavkom \texttt{case}}

V primeru, da je izraz, ki ga računamo v \texttt{case} izrazu (angl. scrutinee) pomnilniški naslov, ki kaže na konstruktor $C$, potem se izvede telo veje s konstruktorjem $C$, pri katerem se vrednosti parametrov $x_1, \dots x_n$ zamenjajo z argumenti $a_1, \dots, a_n$ (pravilo \ref{eq:stg-casecon}).

\begin{equation}
	\linfer[casecon]{}{
		\texttt{case} \enspace v \enspace \texttt{of} \enspace \{ \dots; C \, x_1, \dots, x_n \to e; \dots \}; \, s; \, H[v \mapsto \textsc{con}(C \, a_1 \dots a_n)]
	}{
		e[a_1 / x_1 \dots a_n / x_n]; \, s; \, H
	}
\label{eq:stg-casecon}
\end{equation}

Če se noben izmed konstruktorjev v algebraičnih alternativah ne ujema s konstruktorjem na naslovu $v$, ali če je $v$ konstanta, potem se izvede privzeta alternativa (pravilo \ref{eq:stg-caseany}). Pri tem se v telesu alternative parameter $x$ zamenja z $v$. V STG jeziku so vrednosti objekti \textsc{fun}, \textsc{pap} in \textsc{con}.

\begin{equation}
	\infer[caseany]{
		(\text{$v$ je konstanta}) \lor (\text{$H[v]$ je vrednost, ki ne ustreza nobeni drugi alternativi})
	}{
		\texttt{case} \enspace v \enspace \texttt{of} \enspace \{ \dots; x \to e \}; \, s; \, H
	}{
		e[v/x]; \, s; \, H
	}
 \label{eq:stg-caseany}
\end{equation}

Pravilo \ref{eq:stg-case} začne izračun \texttt{case} izraza. Najprej se začne računati izraz $e$, na sklad pa se potisne kontinuacija \texttt{case} izraza, ki se bo izvedla, ko se bo vrednost $e$ do konca izračunala.

\begin{equation}
	\infer[case]{}{
		\texttt{case} \enspace e \enspace \texttt{of} \enspace \{ \dots \}; \, s; \, H
	}{
		e; \, (\texttt{case} \; \bullet \; \texttt{of} \; \{ \dots \}) : s; \, H
	}
\label{eq:stg-case}
\end{equation}

Zadnje pravilo \ref{eq:stg-ret} se izvede, ko se izraz v \texttt{case} izrazu do konca izračuna v konstanto ali kazalec na objekt na kopici. Glede na tip objekta na katerega kaže spremenljivka $v$, se bo nato izvedlo eno izmed pravil \ref{eq:stg-casecon} ali \ref{eq:stg-caseany}.

\begin{equation}
	\infer[ret]{
		(\text{$v$ je konstanta}) \lor (\text{$H[v]$ je vrednost})
	}{
		v; \, (\texttt{case} \; \bullet \; \texttt{of} \; \{ \dots \}) : s; \, H
	}{
		\texttt{case} \enspace v \enspace \texttt{of} \enspace \{ \dots \}; \, s; \, H
	}
\label{eq:stg-ret}
\end{equation}

Pomembno je še omeniti, da se pri izvajanju nikoli ne brišejo vezave objektov na kopici. V primeru pravila \ref{eq:stg-casecon}, se tako iz kopice ne izbriše vezava $v \mapsto \textsc{con}(C \, a_1 \dots a_n)$. Brisanje elementov iz kopice je implementirano s pomočjo avtomatičnega čistilca pomnilnika.

\subsubsection{Zakasnitve in njih posodobitve}

Pravili \ref{eq:stg-thunk} in \ref{eq:stg-update} sta namenjeni izračunu zakasnitev. Če je trenutni izraz ravno kazalec na zakasnitev, potem se na sklad doda nova kontinuacija oziroma posodobitveni okvir $\textit{Upd} \; t \; \bullet$, ki kaže na spremenljivko $x$, ki se bo po izračunu posodobila. Prav tako se na kopici objekt zamenja z \textsc{blackhole}. Če med izračunom vrednosti izraza $e$ abstraktni stroj ponovno naleti na spremenljivko $x$, lahko predpostavi, da program vsebuje cikel. Ker nobeno izmed pravil operacijske semantike na levi strani ne vsebuje objekta \textsc{blackhole}, se v primeru ciklov tako izračun zaustavi.

\begin{equation}	
\infer[thunk]{}{
	x; \, s; \, H[x \mapsto \textsc{thunk}(e)]
}{
	e; \, (\textit{Upd} \; t \; \bullet) : s; \, H[x \mapsto \textsc{blackhole}]
}
\label{eq:stg-thunk}
\end{equation}

Pri pravilu \ref{eq:stg-update} se izvede posodobitev zakasnitve na kopici, pri čemer se spremenljivko $x$ preusmeri na objekt na katerega kaže spremenljivka $y$.

\begin{equation}
\infer[update]{
	\text{$H[y]$ je vrednost}
}{
	y; \, (\textit{Upd} \; x \; \bullet) : s; \, H
}{
	y; \, s; \, H[x \mapsto H[y]
}
\label{eq:stg-update}
\end{equation}

V dejanski implementaciji se pri posodobitvi na pomnilniku prejšnji ob\-je\-kt prepiše s preusmeritvijo. To je objekt, ki je (podobno kot ostali objekti v STG jeziku glede na sliko \ref{fig:shema-stg-objekt}) sestavljen iz kazalca na metapodatke objekta in še enega dodatnega polja - kazalca na drug objekt. To pa tudi pomeni, da morajo imeti vsi objekti na kopici prostora vsaj za 2 kazalca (2 polji), saj bi sicer pri posodobitvi prišlo do prekoračitve kopice oziroma prepisovanja naslednjega objekta.

\subsubsection{Klici funkcij z znano mestnostjo}

Pri pravilu \ref{eq:stg-knowncall} gre za uporabo funkcije z mestnostjo $n$ nad natanko $n$ argumenti. Izvede se telo funkcije, v katerem so vsi parametri zamenjani z vrednostmi argumentov, sklad in kopica pa ostajata nespremenjena.

\begin{equation}
	\linfer[knowncall]{}{
		f^n \, a_1 \dots a_n; \, s; \, H[f \mapsto \textsc{fun}(x_1 \dots x_n \to e)]
	}{
		e[a_1 / x_1 \dots a_n / x_n]; \, s; \, H
	}
\label{eq:stg-knowncall}
\end{equation}

Pravilo \ref{eq:stg-primop} je zelo podobno pravilu \ref{eq:stg-knowncall}, le da se vrednost primitivne operacije izračuna neposredno v okolju za izvajanje. Ker so primitivne operacije po definiciji v STG jeziku vedno nasičene, zanje niso potrebna nobena dodatna pravila.

\begin{equation}
	\infer[primop]{
		a = \oplus \, a_1 \dots a_n
	}{
		\oplus \, a_1 \dots a_n; \, s; \, H
	}{
		a; \, s; \, H
	}
\label{eq:stg-primop}
\end{equation}

Pri pravilih \ref{eq:stg-knowncall} in \ref{eq:stg-primop} gre za aplikacijo funkcije z znano mestnostjo $n$ na natanko $n$ argumentov. Kaj pa če argumentov ni dovolj ali pa jih je preveč? V primeru, da je argumentov premalo, gre za delno aplikacijo funkcije. Če pa je argumentov preveč, je potrebno funkcijo najprej aplicirati na ustrezno število argumentov. Ko se bo aplikacija do konca izvedla, bo rezultat funkcija, ki jo apliciramo na preostanek argumentov.

\subsubsection{Klici funkcij z neznano mestnostjo}

Pravilo \ref{eq:stg-exact} je zelo podobno pravilu \ref{eq:stg-knowncall}, le ga gre v tem primeru za funkcijo neznane mestnosti. V tem primeru je število argumentov shranjeno v tabeli metapodatkov objekta \textsc{fun} na kopici, kot bomo videli v poglavju \ref{sec:abstraktni-stg-stroj}.

\begin{equation}
	\linfer[exact]{}{
		f^\bullet \, a_1 \dots a_n ; \, s; \, H[f \mapsto \textsc{fun}(x_1 \dots x_n \to e)]
	}{
		e[a_1 / x_1 \dots a_n / x_n]; \, s; \, H
	}
\label{eq:stg-exact}
\end{equation}

Pravili \ref{eq:stg-callk} in \ref{eq:stg-pap2} se ukvarjata z aplikacijami funkcij, pri katerih je število argumentov različno od njihove mestnosti. Če je število argumentov $m$ več kot parametrov $n$ (pravilo \ref{eq:stg-callk}), potem se prvih $n$ argumentov porabi pri aplikaciji funkcije, vrednosti preostalih argumentov $a_{n + 1}, \dots, a_m$ pa se doda na sklad. Ko se aplikacija do konca izračuna, bo rezultat še ena funkcija, ki je uporabljena na preostalih argumentih (pravilo \ref{eq:stg-retfun}). Če argumentov pri aplikaciji ni dovolj, se na kopici ustvari nov objekt \textsc{pap}, ki predstavlja delno aplikacijo.

\begin{equation}
	\linfer[callk]{
		m > n
	}{
		f^k \, a_1 \dots a_m ; \, s; \, H[f \mapsto \textsc{fun}(x_1 \dots x_n \to e)]
	}{
		e[a_1 / x_1 \dots a_n / x_n]; \, (\bullet \; a_{n + 1} \dots a_m) : s; \, H
	}
\label{eq:stg-callk}
\end{equation}
\vspace{1em}
\begin{equation}
	\linfer[pap2]{
		m < n, \text{$p$ je sveža spremenljivka}
	}{
		f^k \, a_1 \dots a_m ; \, s; \, H[f \mapsto \textsc{fun}(x_1 \dots x_n \to e)]
	}{
		p; \, s; \, H[p \mapsto \textsc{pap}(f \, a_1 \dots a_m)]
	}
\label{eq:stg-pap2}
\end{equation}

Pravilo \ref{eq:stg-tcall} pokrije možnost, da funkcija še ni izračunana, torej da je zakasnitev. V tem primeru se prične računanje zakasnitve $f$, obenem pa se na sklad doda kontinuacija $(\bullet \; a_1 \dots a_m)$, ki bo sprožila klic funkcije, ko bo vrednost zakasnitve do konca izračunana.

\begin{equation}
	\linfer[tcall]{}{
		f^\bullet \, a_1 \dots a_m ; \, s; \, H[f \mapsto \textsc{thunk}(e)]
	}{
		f; \, (\bullet \; a_1 \dots a_m) : s; \, H
	}
\label{eq:stg-tcall}
\end{equation}

Pri pravilu \ref{eq:stg-pcall} gre za uporabo delne aplikacije nad preostankom argumentov. V tem primeru se funkcija $g$ uporabi nad vsemi $m$ argumenti. Ker je $g$ zaradi definicije v poglavju \ref{sec:stg-definicija} zagotovo funkcija, se bo v naslednjem koraku zagotovo zgodilo eno izmed pravil \ref{eq:stg-exact}, \ref{eq:stg-callk} ali \ref{eq:stg-pap2}.

\begin{equation}
	\linfer[pcall]{}{
		f^k \, a_{n + 1} \dots a_m ; \, s; \, H[f \mapsto \textsc{pap}(g \, a_1 \dots a_n)]
	}{
		g^\bullet \, a_1 \dots a_n \, a_{n + 1} \dots a_m; \, s; \, H
	}
\label{eq:stg-pcall}
\end{equation}

Zadnje pravilo \ref{eq:stg-retfun}, ki deluje na podoben način kot pravilo \ref{eq:stg-ret} pri \ref{eq:stg-case} izrazih. Kadar se izraz izračuna v kazalec, ki kaže na funkcijo \textsc{fun} ali delno aplikacijo \textsc{pap} in je na skladu kontinuacija z argumenti, se ustrezno izvede aplikacija funkcije nad argumenti.

\begin{equation}
	\infer[retfun]{
		(\text{$H[f]$ je \textsc{fun}(\dots)}) \lor (\text{$H[f]$ je \textsc{pap}(\dots)})
	}{
		f; \, (\bullet \; a_1 \dots a_n) : s; \, H
	}{
		f^\bullet \, a_1 \dots a_n; \, s; \, H
	}
\label{eq:stg-retfun}
\end{equation}

V nadaljevanju bomo na kratko opisali delovanje abstraktnega STG stroja in predstavitev objektov na pomnilniku.

\section{Abstraktni STG stroj}
\label{sec:abstraktni-stg-stroj}

Slika \ref{fig:shema-stg-objekt} prikazuje strukturo STG objektov na pomnilniku. Vsak objekt je sestavljen iz kazalca na tabelo metapodatkov in polj z vsebino objekta (angl. payload). V poljih z vsebino so shranjene proste spremenljivke, ki so lahko ali konstantne vrednosti ali pa kazalci, ki kažejo na druge objekte na kopici. Ker STG jezik podpira 6 različnih vrst objektov (tj. 5 osnovnih in preusmeritev \object{indirection}), se v tabeli metapodatkov hrani polje z vrsto objekta. Različni objekti nosijo različno vsebino. Objekt $\textsc{con}(C \; a_1 \, \dots \, a_n)$ je npr. predstavljen s strukturo, ki ima na prvem mestu kazalec do metapodatkov objekta tipa \object{con}, vsebina pa ima $n$ polj, z vrednostmi argumentov $a_1, \dots, a_n$ (ki so, kot smo videli v poglavju \ref{sec:stg-definicija} atomarne vrednosti). 

Tabela metapodatkov vsebuje dodatne informacije, ki jih potrebujemo pri izvajanju programa. Sestavlja jo kazalec na kodo, vrsta objekta, struktura objekta in dodatna polja, ki so specifična vrsti objekta. Kazalec na kodo kaže na kodo, ki se izvede pri evalvaciji objekta. Vrsta objekta je oznaka (npr. kar celoštevilska vrednost), ki razlikuje šest različnih vrst objektov (\object{thunk}, \object{fun}, \object{pap}, \object{con}, \object{blackhole} in \object{indirection}) med seboj. Struktura objekta vsebuje podatke o tem katera polja predstavljajo kazalce in katera polja dejanske vrednosti oziroma konstante. V STG stroju se uporablja za avtomatično čiščenje pomnilnika, saj je pri tem v prvem koraku potrebno ugotoviti kateri objekti so živi, to pa sistem stori tako, da preko kazalcev obišče vse objekte. Vsak objekt vsebuje še dodatna polja, ki pa so odvisna od vrste objekta. Tako funkcija \object{fun} npr. vsebuje njeno mestnost, medtem ko hrani konstruktor \object{con} njegovo oznako.

\begin{figure*}[ht]
	\centering
	\begin{tikzpicture}[y=-1cm,scale=0.9]
		\draw (0,0) rectangle ++(2,0.75) node[midway] (info-kazalec) {};
		\draw (2,0) rectangle ++(4,0.75) node[midway] {Vsebina};
		
		\draw (1.5,1.25) rectangle ++(4,0.75) node[midway] {Kazalec na kodo};
		\draw (1.5,2) rectangle ++(4,0.75) node[midway] {Vrsta objekta};
		\draw (1.5,2.75) rectangle ++(4,0.75) node[midway] {Struktura objekta};
		\draw (1.5,3.5) rectangle ++(4,1.5) node[midway,align=center,font=\linespread{0.8}\selectfont] {Polja specifična\\vrsti objekta};
		
		\node at (1.5+7,1.25+0.375) {Koda};
		
		\draw[{Circle}-Latex] (1,0.375) -- (1, 1.25+0.375) --(1.5, 1.25+0.375);
		\draw[{Circle}-Latex] (1.5+4-0.25,1.25+0.375) -- (1.5+6,1.25+0.375);
		\draw [decorate,decoration={brace,amplitude=8pt,mirror,raise=3em}]
		(1.5,1.25+0.375) -- (1.5,1.25+3.75-0.375) node[midway,xshift=-7em,align=center,font=\linespread{0.8}\selectfont]{Metapodatki\\objekta};
	\end{tikzpicture}
	\caption{Struktura STG objektov na pomnilniku}
	\label{fig:shema-stg-objekt}
\end{figure*}

% Razlika med modeloma potisni / vstopi in izračunaj / apliciraj
Pri implementaciji nestroge semantike v programskih jezikih obstajata dva glavna pristopa, ki se razlikujeta glede na način, kako se obravnavajo klici funkcij in njihovi argumenti. Modela se razlikujeta predvsem v prelaganju odgovornosti za obravnavo argumentov med klicočo in klicano funkcijo, kar vpliva na način obdelave funkcijskih klicev ter na učinkovitost izvajanja jezika.

\begin{itemize}
	\itemsep 0em
	\item Model \textbf{potisni / vstopi} (angl. push / enter) najprej potisne vse argumente na sklad in nato \textit{vstopi} v funkcijo. Funkcija je odgovorna za preverjanje ali je na skladu dovolj argumentov. Če jih ni, potem mora sestaviti delno aplikacijo na kopici in končati izvajanje. Če pa je argumentov preveč, jih mora funkcija s sklada vzeti le ustrezno število, ostale argumente pa pustiti, saj jih bo uporabila naslednja funkcija, tj. funkcija v katero se bo evalviral trenutni izraz.
	\item Model \textbf{izračunaj / uporabi} (angl. eval / apply) klicatelj najprej evalvira funkcijo in jo nato aplicira na ustrezno število argumentov. Pri aplikaciji je potrebno zahtevano število argumentov funkcije pridobiti med izvajanjem programa iz funkcijske ovojnice.
\end{itemize}

Modela se razlikujeta glede na prelaganje odgovornosti za obravnavo argumentov. Pri modelu potisni / vstopi mora klicana funkcija preveriti ali je na skladu dovolj argumentov. Pri modelu izračunaj in apliciraj pa mora klicoča funkcija med izvajanjem pogledati v funkcijsko ovojnico in jo poklicati z ustreznim številom argumentov.

% Katero različico trenutno uporablja Haskell
Izkaže se, da je hitrejši model izračunaj / apliciraj~\cite{marlow2004making}, zato je ta model tudi uporabljen v Haskellovem prevajalniku GHC, kot bomo videli v poglavju \ref{ch:problemi} pa smo ga uporabili tudi pri naši implementaciji prevajalnika.

% Kaj se spremeni, če dodamo enega in drugega

\section{Primer}
\komentar{Manjka primer izvajanja manjšega programa v STG jeziku.}
