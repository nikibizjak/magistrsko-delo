\chapter{Uvod}
\label{ch:uvod}

Pomnilnik je dandanes kljub uvedbi pomnilniške hierarhije še vedno eden izmed najpočasnejših delov računalniške arhitekture. Učinkovito upravljanje s pomnilnikom je torej ključnega pomena za učinkovito izvajanje programov. Upravljanje s pomnilnikom v grobem ločimo na ročno in avtomatično~\cite{jones2023garbage}. Pri ročnem upravljanju s pomnilnikom programski jezik vsebuje konstrukte za dodeljevanje in sproščanje pomnilnika. Odgovornost upravljanja s pomnilnikom leži na programerju, zato je ta metoda podvržena človeški napaki. Pogosti napaki sta puščanje pomnilnika (angl. memory leaking), pri kateri dodeljen pomnilnik ni sproščen, in viseči kazalci (angl. dangling pointers), ki kažejo na že sproščene in zato neveljavne dele pomnilnika~\cite{jones2023garbage}.

Pri avtomatičnem upravljanju s pomnilnikom zna sistem sam dodeljevati in sproščati pomnilnik. Tukaj ločimo posredne in neposredne metode. Ena izmed neposrednih metod je npr. štetje referenc~\cite{collins1960method}, pri kateri za vsak objekt na kopici hranimo metapodatek o številu kazalcev, ki se sklicujejo nanj. V tem primeru moramo ob vsakem spreminjanju referenc zagotavljati še ustrezno posodabljanje števcev, kadar pa število kazalcev pade na nič, objekt izbrišemo iz pomnilnika. Posredne metode, npr. označi in pometi~\cite{mccarthy1960recursive}, ne posodabljajo metapodatkov na pomnilniku ob vsaki spremembi, temveč se izvedejo le, kadar se prekorači velikost kopice. Algoritem pregleda kopico in ugotovi, na katere objekte ne kaže več noben kazalec ter jih odstrani. Nekateri algoritmi podatke na kopici tudi defragmentirajo in s tem zagotovijo boljšo lokalnost ter s tem boljše predpomnjenje~\cite{fenichel1969lisp}.

Avtomatično čiščenje pomnilnika pa ima tudi svoje probleme. Štetje referenc v primeru pomnilniških ciklov privede do puščanja pomnilnika, metoda označi in pometi pa nedeterministično zaustavi izvajanje glavnega programa in tako ni primerna za časovno-kritične (angl. real-time) aplikacije. Kot alternativa obem načinom upravljanja s pomnilnikom sistemski programski jezik Rust implementira model lastništva~\cite{klabnik2023rust}. Med \textit{prevajanjem} zna s posebnimi pravili zagotoviti, da se pomnilnik objektov na kopici avtomatično sprosti, kadar jih program več ne potrebuje. To pa zna storiti brez čistilca pomnilnika in brez eksplicitnega dodeljevanja in sproščanja pomnilnika, zato zagotavlja predvidljivo sproščanje pomnilnika.

V magistrskem delu se bomo primarno ukvarjali s STG jezikom. Operacijska semantika tega veleva, da so vsi izrazi v izvorni kodi v pomnilniku predstavljeni kot zaprtja. Jezik vsebuje izraz \texttt{let}, ki na kopici ustvari novo zaprtje, izbirni izraz \texttt{case} pa šele dejansko izračuna njegovo vrednost. Jezik STG za čiščenje zaprtij iz kopice uporablja generacijski čistilec pomnilnika~\cite{jones1992implementing, marlow2004making}. 

Cilj magistrske naloge je pripraviti simulator STG stroja, nato pa spremeniti STG jezik tako, da bo namesto avtomatičnega čistilca pomnilnika uporabljal model lastništva po zgledu programskega jezika Rust. Zanimalo nas bo, kakšne posledice to v STG stroj prinese, kakšne omejitve se pri tem pojavijo ter do kakšnih problemov lahko pri tem pride. Zavedati se moramo, da obstaja možnost, da koncepta lastništva ni mogoče vpeljati v STG stroj brez korenitih sprememb zasnove stroja samega - v tem primeru bomo podali analizo, zakaj lastništva v STG stroj ni mogoče vpeljati.

\section{Upravljanje s pomnilnikom}

Pomnilnik je dandanes, kljub uvedbi pomnilniške hierarhije, še vedno eden izmed najpočasnejših delov računalniške arhitekture. Učinkovito upravljanje s pomnilnikom je torej ključnega pomena za učinkovito izvajanje programov. Upravljanje s pomnilnikom v grobem ločimo na ročno in avtomatično~\cite{jones2023garbage}. Pri ročnem upravljanju s pomnilnikom programski jezik vsebuje konstrukte za dodeljevanje in sproščanje pomnilnika. Odgovornost upravljanja s pomnilnikom leži na programerju, zato je ta metoda podvržena človeški napaki. Pogosti napaki sta puščanje pomnilnika (angl. memory leaking), pri kateri dodeljen pomnilnik ni sproščen in viseči kazalci (angl. dangling pointers), ki kažejo na že sproščene in zato neveljavne dele pomnilnika~\cite{jones2023garbage}.

Pri ročnem upravljanju pomnilnika, kot ga poznamo npr. pri programskem jeziku C, pride pogosto do dveh vrst napak~\cite{jones2023garbage}:

\begin{itemize}
	\itemsep 0em
	\item uporaba po sproščanju (angl. use-after-free), pri kateri program dostopa do bloka pomnilnika, ki je že bil sproščen in
	\item dvojno sproščanje (angl. double free), pri katerem se skuša isti blok pomnilnika sprostiti dvakrat.
\end{itemize}

V obeh primerih pride do nedefiniranega obnašanja sistema za upravljanje s pomnilnikom (angl. memory management system). Ob uporabi po sproščanju lahko pride npr. do dostopanja do pomnilniškega naslova, ki ni več v lasti trenutnega procesa in posledično do sesutja programa. V primeru dvojnega sproščanja pa lahko pride do okvare delovanja sistema za upravljanje s pomnilnikom, kar lahko privede do dodeljevanja napačnih naslovov ali prepisovanja obstoječih podatkov na pomnilniku.

Druga možnost upravljanja s pomnilnikom je avtomatično upravljanje pomnilnika, pri katerem zna sistem sam dodeljevati in sproščati pomnilnik. Pri avtomatičnem upravljanju s pomnilnikom nikoli ne pride do visečih kazalcev, saj je objekt na kopici odstranjen le v primeru, da nanj ne kaže kazalec iz nobenega drugega živega objekta. Ker pri je avtomatičnem upravljanju sistem za upravljanje s pomnilnikom edina komponenta, ki sprošča pomnilnik, je tudi zagotovljeno, da nikoli ne pride do dvojnega sproščanja. Glede na način delovanja ločimo posredne in neposredne metode. Pri neposrednih metodah zna sistem za upravljanje s pomnilnikom prepoznati živost objekta neposredno iz zapisa objekta na pomnilniku, pri posrednih metodah pa sistem s sledenjem kazalcem prepozna vse žive objekte, vse ostalo pa smatra za nedostopne oziroma neuporabljene objekte in jih ustrezno počisti.

Ena izmed neposrednih metod je npr. štetje referenc~\cite{collins1960method}, pri kateri za vsak objekt na kopici hranimo metapodatek o številu kazalcev, ki se sklicujejo nanj. V tem primeru moramo ob vsakem spreminjanju referenc zagotavljati še ustrezno posodabljanje števcev, kadar pa število kazalcev pade na nič, objekt izbrišemo iz pomnilnika. Ena izmed slabosti štetja referenc je, da mora sistem ob vsakem prirejanju vrednosti v spremenljivko posodobiti še števec v pomnilniku, kar privede do povečanja števila pomnilniških dostopov. Prav tako pa metoda ne deluje v primeru pomnilniških ciklov, kjer dva ali več objektov kažeta drug na drugega. V tem primeru števec referenc nikoli ne pade na nič, kar pomeni da pomnilnik nikoli ni ustrezno počiščen.

Posredne metode, npr. označi in pometi~\cite{mccarthy1960recursive}, ne posodabljajo metapodatkov na pomnilniku ob vsaki spremembi, temveč se izvedejo le, kadar se prekorači velikost kopice. Algoritem pregleda kopico in ugotovi, na katere objekte ne kaže več noben kazalec ter jih odstrani. Nekateri algoritmi podatke na kopici tudi defragmentirajo in s tem zagotovijo boljšo lokalnost ter s tem boljše predpomnjenje~\cite{fenichel1969lisp}. Problem metode označi in pometi pa je njeno nedeterministično izvajanje, saj programer oziroma sistem ne moreta predvideti, kdaj se bo izvajanje glavnega programa zaustavilo, in tako ni primerna za časovno-kritične (angl. real-time) aplikacije. Za razliko od štetja referenc pa zna algoritem označi in pometi počistiti tudi pomnilniške cikle in se ga zato včasih uporablja v kombinaciji s štetjem referenc. Programski jezik Python za čiščenje pomnilnika primarno uporablja štetje referenc, periodično pa se izvede še korak metode označi in pometi, da odstrani pomnilniške cikle, ki jih prva metoda ne zmore~\cite{van2007python}.
\section{Rustov model lastništva}

Programski jezik Rust je namenjen nizkonivojskemu programiranju, tj. programiranju sistemske programske opreme. Kot tak mora omogočati hitro in predvidljivo sproščanje pomnilnika, zato avtomatični čistilnik pomnilnika ne pride v poštev. Rust namesto tega implementira model lastništva~\cite{klabnik2023rust}, pri katerem zna med \textit{prevajanjem} s posebnimi pravili zagotoviti, da se pomnilnik objektov na kopici avtomatično sprosti, kadar jih program več ne potrebuje. Po hitrosti delovanja se tako lahko kosa s programskim jezikom C, pri tem pa zagotavlja varnejše upravljanje s pomnilnikom kot C.

Rust doseže varnost pri upravljanju pomnilnika s pomočjo principa izključitve~\cite{Jung}. V poljubnem trenutku za neko vrednost na pomnilniku velja natanko ena izmed dveh možnosti:

\begin{itemize}
	\itemsep 0em
	\item Vrednost lahko mutiramo preko \textit{natanko enega} unikatnega kazalca
	\item Vrednost lahko beremo preko poljubno mnogo kazalcev
\end{itemize}

\subsection{Premik}

Eden izmed najpomembnejših konceptov v Rustu je lastništvo. Ta zagotavlja, da si vsako vrednost na pomnilniku lasti natanko ena spremenljivka. Kadar gre ta izven dosega (angl. out-of-scope), lahko tako Rust ustrezno počisti pomnilnik. Ob klicu funkcije se lastništvo nad argumenti prenese v funkcijo, ta pa postane odgovorna za čiščenje pomnilnika. Naslednji primer prikazuje program, ki se v Rustu ne prevede zaradi težav z lastništvom.

\begin{rust-failure}
let number = Complex(0, 1);
let a = number;
let b = number;  // Napaka: use of moved value: `number`
\end{rust-failure}

Spremenljivka \texttt{a} prevzame lastništvo nad vrednostjo \texttt{number}. Pravimo tudi, da je bila vrednost \texttt{number} \textit{premaknjena} v spremenljivko \texttt{a}. Ob premiku postane spremenljivka \texttt{number} neveljavna, zato pri ponovnem premiku v spremenljivko \texttt{b}, prevajalnik javi napako.

\begin{rust-failure}
let number = Complex(1, 0);
f(number);
let x = number;  // Napaka: use of moved value: `number`
\end{rust-failure}

V zgornjem primeru se spremenljivka \texttt{number} pojavi kot argument funkciji \texttt{f}. Lastništvo spremenljivke \texttt{number} se ob klicu funkcije prenese v funkcijo in ta postane tudi odgovorna za čiščenje njenega pomnilnika. Ob ponovnem premiku v spremenljivko \texttt{x}, prevajalnik javi napako.

\subsection{Izposoja}

Drugi koncept, ki ga definira Rust je \textit{izposoja}. Ta omogoča \textit{deljenje} (angl. aliasing) vrednosti na pomnilniku. Izposoje so lahko spremenljive (angl. mutable) \texttt{\&mut x} ali nespremenljive (angl. immutable) \texttt{\&x}. V danem trenutku je lahko ena spremenljivka izposojena nespremenljivo oziroma samo za branje (angl. read-only) večkrat, spremenljivo pa natanko enkrat.

Naslednji primer se v Rustu uspešno prevede, ker sta obe izposoji nespremenljivi. Vrednosti spremenljivk \texttt{a} in \texttt{b} lahko le beremo, ne moremo pa jih spreminjati.

\begin{rust-success}
let number = Complex(2, 1);
let a = &number;
let b = &number;
\end{rust-success}

V naslednjem primeru skušamo ustvariti dve mutable referenci na spremenljivko \texttt{number}. Zaradi principa izključevanja to ni mogoče, zato prevajalnik javi napako.

\begin{rust-failure}
let mut number = Complex(1, 2);
let a = &mut number;
let b = &mut number;  // Napaka: cannot borrow `number` as mutable
                      // more than once at a time
\end{rust-failure}

Prav tako ni veljavno ustvariti mutabilne reference na spremenljivko, dokler nanjo obstaja kakršnakoli druga referenca.

\begin{rust-failure}
let mut number = Complex(0, 0);
let a = &number;
let b = &mut number;  // Napaka: cannot borrow `number` as mutable
                      // because it is also borrowed as immutable
\end{rust-failure}

\subsection{Delna izposoja}

\begin{rust-success}
let number = Complex(2, 1);
let a = number.1;
let b = number.2;
\end{rust-success}

\subsection{Ponovna izposoja}

\begin{rust-success}
let number = Complex(2, 1);
let a = &number;
let b = *a;
\end{rust-success}

% Non-lexical lifetimes
V trenutni različici Rusta 2018 bi se vsi zgornji primeri vseeno uspešno prevedli. Projekt Polonius~\cite{Matsakis_2018, Matsakis_et_al_2022} je v programski jezik namreč uvedel neleksikalne življenjske dobe (angl. non-lexical lifetimes), ki preverjanje izposoj in premikov izvaja nad CFG namesto nad abstraktnim sintaktičnim drevesom programa~\cite{Oxide, StackedBorrows}. Prevajalnik zna s pomočjo neleksikalnih življenskih dob dokazati, da sta dve zaporedni mutable izposoji varni, če ena izmed njih ni nikjer uporabljena. Na tak način Rust zagotovi bolj drobnozrnat (angl. fine grained) pogled na program, saj prevajalnik vrača napake ob manj veljavnih programih. Za vse zgornje primere torej predpostavljamo, da tako premaknjene, kot tudi izposojene spremenljivke spremenljivke v nadaljevanju še nekje uporabljene.

% Formalizacije Rusta
Kljub temu, da Rust v svoji dokumentaciji~\cite{klabnik2023rust} zagotavlja, da je njegov model upravljanja s pomnilnika varen, pa njegovi razvijalci niso nikoli uradno formalizirali niti njegove operacijske semantike, niti sistema tipov, niti modela za prekrivanje (angl. aliasing model). V literaturi se je tako neodvisno uveljavilo več modelov, ki skušajo čimbolj natančno formalizirati semantiko premikov in izposoj. Model Patina~\cite{reed2015patina} formalizira delovanje analize izposoj v Rustu in dokaže, da velja izrek o varnosti (angl. soundness) za safe podmnožico jezika Rust. Model RustBelt~\cite{10.1145/3158154} poda prvi formalen in strojno preverjen dokaz o varnosti za jezik, ki predstavlja realistično podmnožico jezika Rust. Model Stacked borrows~\cite{StackedBorrows} definira operacijsko semantiko za dostopanje do pomnilnika v Rustu in model za prekrivanje (angl. aliasing model) in ga strojno dokaže. Model Oxide~\cite{Oxide} je formalizacija Rustovega sistema tipov in je tudi prva semantika, ki podpira tudi neleksikalne življenjske dobe. 