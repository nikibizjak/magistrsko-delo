\chapter{Uvod}
\label{ch:uvod}

Pomnilnik je dandanes kljub uvedbi pomnilniške hierarhije še vedno eden izmed najpočasnejših delov računalniške arhitekture. Učinkovito upravljanje s pomnilnikom je torej ključnega pomena za učinkovito izvajanje programov. Upravljanje s pomnilnikom v grobem ločimo na ročno in avtomatično~\cite{jones2023garbage}. Pri ročnem upravljanju s pomnilnikom programski jezik vsebuje konstrukte za dodeljevanje in sproščanje pomnilnika. Odgovornost upravljanja s pomnilnikom leži na programerju, zato je ta metoda podvržena človeški napaki. Pogosti napaki sta puščanje pomnilnika (angl. memory leaking), pri kateri dodeljen pomnilnik ni sproščen, in viseči kazalci (angl. dangling pointers), ki kažejo na že sproščene in zato neveljavne dele pomnilnika~\cite{jones2023garbage}.

Pri avtomatičnem upravljanju s pomnilnikom zna sistem sam dodeljevati in sproščati pomnilnik. Tukaj ločimo posredne in neposredne metode. Ena izmed neposrednih metod je npr. štetje referenc~\cite{collins1960method}, pri kateri za vsak objekt na kopici hranimo metapodatek o številu kazalcev, ki se sklicujejo nanj. V tem primeru moramo ob vsakem spreminjanju referenc zagotavljati še ustrezno posodabljanje števcev, kadar pa število kazalcev pade na nič, objekt izbrišemo iz pomnilnika. Posredne metode, npr. označi in pometi~\cite{mccarthy1960recursive}, ne posodabljajo metapodatkov na pomnilniku ob vsaki spremembi, temveč se izvedejo le, kadar se prekorači velikost kopice. Algoritem pregleda kopico in ugotovi, na katere objekte ne kaže več noben kazalec ter jih odstrani. Nekateri algoritmi podatke na kopici tudi defragmentirajo in s tem zagotovijo boljšo lokalnost ter s tem boljše predpomnjenje~\cite{fenichel1969lisp}.

Avtomatično čiščenje pomnilnika pa ima tudi svoje probleme. Štetje referenc v primeru pomnilniških ciklov privede do puščanja pomnilnika, metoda označi in pometi pa nedeterministično zaustavi izvajanje glavnega programa in tako ni primerna za časovno-kritične (angl. real-time) aplikacije. Kot alternativa obem načinom upravljanja s pomnilnikom sistemski programski jezik Rust implementira model lastništva~\cite{klabnik2023rust}. Med \textit{prevajanjem} zna s posebnimi pravili zagotoviti, da se pomnilnik objektov na kopici avtomatično sprosti, kadar jih program več ne potrebuje. To pa zna storiti brez čistilca pomnilnika in brez eksplicitnega dodeljevanja in sproščanja pomnilnika, zato zagotavlja predvidljivo sproščanje pomnilnika.

V magistrskem delu se bomo primarno ukvarjali s STG jezikom. Operacijska semantika tega veleva, da so vsi izrazi v izvorni kodi v pomnilniku predstavljeni kot zaprtja. Jezik vsebuje izraz \texttt{let}, ki na kopici ustvari novo zaprtje, izbirni izraz \texttt{case} pa šele dejansko izračuna njegovo vrednost. Jezik STG za čiščenje zaprtij iz kopice uporablja generacijski čistilec pomnilnika~\cite{jones1992implementing, marlow2004making}. 

Cilj magistrske naloge je pripraviti simulator STG stroja, nato pa spremeniti STG jezik tako, da bo namesto avtomatičnega čistilca pomnilnika uporabljal model lastništva po zgledu programskega jezika Rust. Zanimalo nas bo, kakšne posledice to v STG stroj prinese, kakšne omejitve se pri tem pojavijo ter do kakšnih problemov lahko pri tem pride. Zavedati se moramo, da obstaja možnost, da koncepta lastništva ni mogoče vpeljati v STG stroj brez korenitih sprememb zasnove stroja samega - v tem primeru bomo podali analizo, zakaj lastništva v STG stroj ni mogoče vpeljati.

\section{Upravljanje s pomnilnikom}

% V magistrskem delu se bomo primarno ukvarjali s STG jezikom. Operacijska semantika tega veleva, da so vsi izrazi v izvorni kodi v pomnilniku predstavljeni kot zaprtja. Jezik vsebuje izraz \texttt{let}, ki na kopici ustvari novo zaprtje, izbirni izraz \texttt{case} pa šele dejansko izračuna njegovo vrednost. Jezik STG za čiščenje zaprtij iz kopice uporablja generacijski čistilec pomnilnika~\cite{jones1992implementing, marlow2004making}. 

% Cilj magistrske naloge je pripraviti simulator STG stroja, nato pa spremeniti STG jezik tako, da bo namesto avtomatičnega čistilca pomnilnika uporabljal model lastništva po zgledu programskega jezika Rust. Zanimalo nas bo, kakšne posledice to v STG stroj prinese, kakšne omejitve se pri tem pojavijo ter do kakšnih problemov lahko pri tem pride. Zavedati se moramo, da obstaja možnost, da koncepta lastništva ni mogoče vpeljati v STG stroj brez korenitih sprememb zasnove stroja samega - v tem primeru bomo podali analizo, zakaj lastništva v STG stroj ni mogoče vpeljati.

Pomnilnik je dandanes kljub uvedbi pomnilniške hierarhije še vedno eden izmed najpočasnejših delov računalniške arhitekture. Učinkovito upravljanje s pomnilnikom je torej ključnega pomena za učinkovito izvajanje programov. Upravljanje s pomnilnikom v grobem ločimo na ročno in avtomatično~\cite{jones2023garbage}. Pri ročnem upravljanju s pomnilnikom programski jezik vsebuje konstrukte za dodeljevanje in sproščanje pomnilnika. Odgovornost upravljanja s pomnilnikom leži na programerju, zato je ta metoda podvržena človeški napaki. Pogosti napaki sta puščanje pomnilnika (angl. memory leaking), pri kateri dodeljen pomnilnik ni sproščen, in viseči kazalci (angl. dangling pointers), ki kažejo na že sproščene in zato neveljavne dele pomnilnika~\cite{jones2023garbage}.

Pri ročnem upravljanju pomnilnika kot ga poznamo npr. pri programskem jeziku C, pride pogosto do dveh vrst napak~\cite{jones2023garbage}:

\begin{itemize}
	\itemsep 0em
	\item viseči kazalci (angl. dangling pointers) so kazalci na pomnilnik, ki je že bil sproščen.
	\item puščanje pomnilnika (angl. memory leak), 
	\item uporaba po sproščanju (angl. use-after-free), pri kateri skušamo dostopati do pomnilnika, ki je bil že prej sproščen in
	\item dvojno sproščanje (angl. double free), pri katerem skušamo dvakrat sprostiti isti pomnilniški naslov.
\end{itemize}

V obeh primerih bo prišlo do nedefiniranega obnašanja sistema za upravljanje s pomnilnikom (angl. memory management system). Ob uporabi po sproščanju lahko pride npr. do dostopanja do pomnilniškega naslova, ki ni več v lasti trenutnega procesa in posledično do sesutja programa. V primeru dvojnega sproščanja pa lahko pride do okvare delovanja sistema za upravljanje s pomnilnikom, kar lahko privede do dodeljevanja napačnih naslovov ali prepisovanja obstoječih podatkov na pomnilniku.

Druga možnost upravljanje s pomnilnikom je avtomatično upravljanje pomnilnika, pri katerem zna sistem sam dodeljevati in sproščati pomnilnik. Tukaj ločimo posredne in neposredne metode.

Ena izmed neposrednih metod je npr. štetje referenc~\cite{collins1960method}, pri kateri za vsak objekt na kopici hranimo metapodatek o številu kazalcev, ki se sklicujejo nanj. V tem primeru moramo ob vsakem spreminjanju referenc zagotavljati še ustrezno posodabljanje števcev, kadar pa število kazalcev pade na nič, objekt izbrišemo iz pomnilnika. Ena izmed slabosti štetja referenc je, da mora sistem ob vsakem prirejanju v spremenljivke posodobiti še števce v pomnilniku, kar privede do povečanja števila pomnilniških dostopov. Prav tako pa metoda ne deluje v primeru pomnilniških ciklov, saj števec referenc nikoli ne pade na nič, kar pomeni da pomnilnik nikoli ni ustrezno počiščen.

Posredne metode, npr. označi in pometi~\cite{mccarthy1960recursive}, ne posodabljajo metapodatkov na pomnilniku ob vsaki spremembi, temveč se izvedejo le, kadar se prekorači velikost kopice. Algoritem pregleda kopico in ugotovi, na katere objekte ne kaže več noben kazalec ter jih odstrani. Nekateri algoritmi podatke na kopici tudi defragmentirajo in s tem zagotovijo boljšo lokalnost ter s tem boljše predpomnjenje~\cite{fenichel1969lisp}. Problem metode označi in pometi pa je njeno nedeterministično izvajanje, saj programer oziroma sistem ne moreta predvideti kdaj se bo izvajanje glavnega programa zaustavilo in tako ni primerna za časovno-kritične (angl. real-time) aplikacije.

% Avtomatično čiščenje pomnilnika pa ima tudi svoje probleme. Štetje referenc v primeru pomnilniških ciklov privede do puščanja pomnilnika, metoda označi in pometi pa nedeterministično zaustavi izvajanje glavnega programa in tako ni primerna za časovno-kritične (angl. real-time) aplikacije. Kot alternativa obem načinom upravljanja s pomnilnikom sistemski programski jezik Rust implementira model lastništva~\cite{klabnik2023rust}. Med \textit{prevajanjem} zna s posebnimi pravili zagotoviti, da se pomnilnik objektov na kopici avtomatično sprosti, kadar jih program več ne potrebuje. To pa zna storiti brez čistilca pomnilnika in brez eksplicitnega dodeljevanja in sproščanja pomnilnika, zato zagotavlja predvidljivo sproščanje pomnilnika.
Programski jezik Rust je namenjen nizkonivojskemu programiranju, tj. programiranju sistemske programske opreme. Kot tak mora omogočati hitro in predvidljivo sproščanje pomnilnika, zato avtomatični čistilnik pomnilnika ne pride v poštev. Rust namesto tega implementira model lastništva~\cite{klabnik2023rust}, pri katerem zna med \textit{prevajanjem} s posebnimi pravili zagotoviti, da se pomnilnik objektov na kopici avtomatično sprosti, kadar jih program več ne potrebuje. Po hitrosti delovanja se tako lahko kosa s programskim jezikom C, pri tem pa zagotavlja varnejše upravljanje s pomnilnikom kot C.

Rust doseže varnost pri upravljanju pomnilnika s pomočjo principa izključitve (angl. exclusion
principle)~\cite{jung2020understanding}. V poljubnem trenutku za neko vrednost na pomnilniku velja natanko ena izmed dveh možnosti:

\begin{itemize}
	\itemsep 0em
	\item Vrednost lahko \textit{spreminjamo} preko \textit{natanko enega} unikatnega kazalca
	\item Vrednost lahko \textit{beremo} preko poljubno mnogo kazalcev
\end{itemize}

V nadaljevanju si bomo na primerih ogledali principa premika in izposoje v jeziku Rust. V vseh primerih bomo uporabljali terko \texttt{Complex}, ki predstavlja kompleksno število z dvema celoštevilskima komponentama in je definirana kot \mintinline{rust}{struct Complex(i32, i32)}.

\section{Premik}

Princip lastništva je eden izmed najpomembnejših konceptov v programskem jeziku Rust. Ta zagotavlja, da si vsako vrednost na pomnilniku lasti natanko ena spremenljivka. Ob prirejanju, tj. izrazu \mintinline{rust}|let x = y|, pride do \textit{premika} vrednosti na katero kaže spremenljivka \var{y} v spremenljivko \var{x}. Po prirejanju postane spremenljivka \var{y} neveljavna in se nanjo v nadaljevanju programa ni več moč sklicevati. Kadar gre spremenljivka, ki si lasti vrednost na pomnilniku izven dosega (angl. out-of-scope), lahko tako Rust ustrezno počisti njen pomnilnik. Naslednji primer prikazuje program, ki se v Rustu ne prevede zaradi težav z lastništvom.

\begin{rust-failure}
let number = Complex(0, 1);
let a = number;
let b = number;  // Napaka: use of moved value: `number`
\end{rust-failure}

Spremenljivka \var{a} prevzame lastništvo nad vrednostjo na katero kaže spremenljivka \var{number}, tj. strukturo \mintinline{rust}|Complex(0, 1)|. Ob premiku postane spremenljivka \var{number} neveljavna, zato pri ponovnem premiku v spremenljivko \var{b}, prevajalnik javi napako.

Pravila lastništva~\cite{klabnik2023rust} so v programskem jeziku Rust sledeča:

\begin{itemize}
	\itemsep 0em
	\item Vsaka vrednost ima lastnika.
	\item V vsakem trenutku je lahko lastnik vrednosti le eden.
	\item Kadar gre lastnik izven dosega (angl. out-of-scope), je vrednost spro\-šče\-na.
\end{itemize}

Rustov model lastništva lahko predstavimo tudi kot graf, v katerem vozlišča predstavljajo spremenljivke oziroma objekte na pomnilniku, povezava med vozlišči $u \to v$ pa označuje, da si spremenljivka $u$ lasti spremenljivko $v$. Ker ima vsaka vrednost natanko enega lastnika, lahko sklepamo, da je tak graf ravno drevo. Kadar gre spremenljivka $v$ izven dosega, lahko Rust počisti celotno poddrevo s korenom $v$ tako, da rekurzivno sprosti pomnilnik za spremenljivke, ki si jih vozlišče $v$ lasti, nato pa počisti še svoj pomnilnik. Preverjanje veljavnost pravil poteka v fazi analize premikov (angl. move check), v program pa se v tej fazi na ustrezna mesta dodajo tudi ukazi za sproščanje pomnilnika.

\subsection{Prenos lastništva pri klicu funkcije}

Kadar je spremenljivka uporabljena v argumentu pri klicu funkcije, je vrednost spremenljivke premaknjena v funkcijo. Če se vrednost spremenljivke uporabi za sestavljanje rezultata funkcije, potem je vrednost ponovno premaknjena iz funkcije in vrnjena klicatelju. Naslednji primer prikazuje identiteto implementirano v Rustu. Pri klicu funkcije, je vrednost spremenljivke \var{number} premaknjena v klicano funkcijo, ker pa je ta uporabljena pri rezultatu funkcije, je vrednost ponovno premaknjena v spremenljivko \var{a} v klicatelju.

\begin{rust-success}
fn identiteta(x: Complex) -> Complex { x }
let number = Complex(0, 1);
let a = identiteta(number);
\end{rust-success}

V naslednjem primeru funkcija \var{prepisi} prevzame lastništvo nad vrednostjo \mintinline{rust}|Complex(0, 1)|, vrne pa novo vrednost \mintinline{rust}|Complex(2, 1)|, pri čemer ne uporabi argumenta funkcije. Funkcija \var{prepisi} je tako odgovorna za čiščenje pomnilnika vrednosti argumenta \var{x}.

\begin{rust-success}
fn prepisi(x: Complex) -> Complex { Complex(2, 1) }
let number = Complex(0, 1);
let a = prepisi(number);
\end{rust-success}

\section{Izposoja}

Drugi koncept, ki ga definira Rust je \textit{izposoja}. Ta omogoča \textit{deljenje} (angl. aliasing) vrednosti na pomnilniku. Izposoje so lahko spremenljive (angl. mutable) \texttt{\&mut x} ali nespremenljive (angl. immutable) \texttt{\&x}. Po principu izključitve, je lahko v danem trenutku ena spremenljivka izposojena nespremenljivo oziroma samo za branje (angl. read-only) večkrat, spremenljivo pa natanko enkrat. Preverjanje veljavnosti premikov se v Rustu izvaja v fazi analize izposoj (angl. borrow check), v kateri se zagotovi, da reference ne živijo dlje od vrednosti na katero se sklicujejo, prav tako pa poskrbi, da je lahko vrednost ali izposojena enkrat spremenljivo ali da so vse izposoje nespremenljive.

Naslednji primer se v Rustu uspešno prevede, ker sta obe izposoji nespremenljivi. Vrednosti spremenljivk \var{a} in \var{b} lahko le beremo, ne moremo pa jih spreminjati. Prav tako je prepovedano spreminjati vrednost spremenljivke \var{number}, dokler nanjo obstaja aktivna izposoja. 

\begin{rust-success}
let number = Complex(2, 1);
let a = &number;
let b = &number;
\end{rust-success}

Zaradi principa izključitve je v Rustu prepovedano ustvariti več kot eno spremenljivo referenco na objekt. V primeru, da bi bilo dovoljeno ustvariti več spremenljivih referenc, bi namreč lahko več niti hkrati spreminjalo in bralo vrednost spremenljivke, kar krši pravila varnosti pomnilnika, saj lahko privede do \komentar{tveganih stanj} (angl. data races). V naslednjem primeru skušamo ustvariti dve spremenljivi referenci, zaradi česar prevajalnik ustrezno javi napako.

\begin{rust-failure}
let mut number = Complex(1, 2);
let a = &mut number;
let b = &mut number;  // Napaka: cannot borrow `number` as mutable
                      // more than once at a time
\end{rust-failure}

Prav tako v Rustu ni veljavno ustvariti spremenljive reference na spremenljivko, dokler nanjo obstaja kakršnakoli druga referenca. V nasprotnem primeru bi lahko bila vrednost v eni niti spremenjena, medtem ko bi jo druga nit brala.

\begin{rust-failure}
let mut number = Complex(0, 0);
let a = &number;
let b = &mut number;  // Napaka: cannot borrow `number` as mutable
                      // because it is also borrowed as immutable
\end{rust-failure}

%\subsubsection{Ponovna izposoja}
%
%\begin{rust-success}
%let number = Complex(2, 1);
%let a = &number;
%let b = *a;
%\end{rust-success}

\subsection{Življenjske dobe}

Kot smo že omenili, analiza izposoj v Rustu zagotovi, da v danem trenutku na eno vrednost kaže le ena spremenljiva referenca ali da so vse reference nanjo nespremenljive. Prav tako pa mora prevajalnik v tej fazi zagotoviti, da nobena referenca ne živi dlje od vrednosti, ki si jo izposoja. To doseže z uvedbo \textit{življenjskih dob} (angl. lifetimes), ki predstavljajo časovne okvirje, v katerih so reference veljavne. Prevajalnik navadno življenjske dobe določi implicitno s pomočjo dosegov (angl. scopes). Vsak ugnezden doseg uvede novo življenjsko dobo, za katero velja, da živi največ tako dolgo kot starševski doseg. Kadar se namreč izvedejo vsi stavki v dosegu, bo pomnilnik vseh spremenljivk, definiranih v dosegu, sproščen. Življenjske dobe označujemo z oznakami \texttt{'a}, \texttt{'b}, \dots, najdaljša življenjska doba pa nosi oznako \texttt{'static}, ki označuje, da je objekt živ tekom celotnega izvajanja programa.

\begin{rust-failure}
let outer;
{
    let inner = 3;
    outer = &inner;  // Napaka: `inner` does not live
                     // long enough
}
\end{rust-failure}

Zgornji primer prikazuje program, ki se v Rustu ne prevede zaradi težav z življenjskimi dobami. Program je sestavljen iz dveh dosegov:
\begin{itemize}
	\itemsep 0em
	\item Spremenljivka \var{outer} živi v zunanjem dosegu, prevajalnik ji dodeli živ\-ljen\-jsko dobo \texttt{'a}.
	\item Nov blok povzroči uvedbo novega dosega z živ\-ljenj\-sko dobo \texttt{'b}, zato prevajalnik spremenljivki \var{inner} določi živ\-ljenj\-sko dobo \texttt{'b}.
\end{itemize}

Ker je notranji doseg uveden znotraj zunanjega dosega, si prevajalnik tudi označi, da je življenjska doba \texttt{'a} vsaj tako dolga kot doba \texttt{'b} (oziroma da mora biti življenjska doba \texttt{'b} največ tako dolga kot \texttt{'a}). To pomeni, da bodo vse spremenljivke, ki so definirane znotraj notranjega dosega živele manj časa od spremenljivk definiranih v zunanjem dosegu. Rust počisti pomnilnik spremenljivke \var{inner}, kadar se notranji doseg konča. V našem primeru bi tako spremenljivka \var{outer} kazala na spremenljivko, ki je že bila uničena, s čemer pa bi v program uvedli viseč kazalec, kar pa krši varnostni model jezika, zato v tem primeru Rust vrne napako.

\subsection{Eksplicitno navajanje življenjskih dob}

V Rustu so eksplicitne življenjske dobe potrebne, kadar prevajalnik ne more samodejno določiti razmerij med življenjskimi dobami referenc v podpisih funkcij, metod ali struktur. Do tega pride pri funkcijah, ki sprejmejo več argumentov in vrnejo rezultat, ki se sklicuje na nekatere izmed njih ali pri strukturah, ki v poljih vsebujejo reference. V teh primerih mora Rust zagotoviti, da so vrnjene reference veljavne vsaj tako dolgo, kot je potrebno, česar pa ne zna izpeljati avtomatično, zato mora programer navesti življenjske dobe eksplicitno.

Naslednji primer prikazuje program, pri katerem pride do napake zaradi nepravilno definiranih življenjskih dob.  Funkcija \var{longest} vrne referenco na daljšo besedo. Po definiciji ta sprejme dve referenci z \emph{enako} življenjsko dobo \texttt{'a} in vrne referenco z življenjsko dobo \texttt{'a}, ki živi tako dolgo kot oba argumenta. V funkciji \var{main}, so definirane tri spremenljivke: \var{first} in \var{result} sta deklarirani v istem bloku in imata zato enako življenjsko dobo, spremenljivka \var{second} pa je deklarirana v ugnezdenem bloku in ima tako krajšo življenjsko dobo. Kadar se notranji blok zaključi, se počisti tudi pomnilnik spremenljivke \var{second}. Toda, v spremenljivko \var{result} se shrani kazalec na daljšo izmed besed, kar je v našem primeru spremenljivka \var{second}, ki pa je izbrisana preden se rezultat izpiše na ekran. Rust zna s pomočjo eksplicitno navedenih življenjskih dob napako tudi odkriti in vrniti napako.

\begin{rust-failure}
fn longest<'a>(first: &'a str, second: &'a str) -> &'a str;
fn main() {
    let first = String::from("Rust");
    let result;
    {
        let second = String::from("Haskell");
        result = longest(first.as_str(), second.as_str());
        // Spremenljivka 'second' gre tukaj izven dosega
    }
    println!("{}", result);
}
\end{rust-failure}

V določenih preprostih primerih zna Rust izpeljati življenjske dobe sam. Predvsem zaradi pisanja krajše kode, Rust namreč podpira izpuščanje živ\-ljenj\-skih dob (angl. lifetime elision) v določenih primerih. Pri tej na podlagi treh pravil prevajalnik samodejno ustvari oziroma dopolni življenjske dobe vhodnih in izhodnih argumentov. Če npr. funkcija kot vhod sprejme referenco in vrne referenco, prevajalnik predpostavlja, da imata obe enaki življenjski dobi. Kadar prevajalnik po pravilih življenjskih dob ne zna izpeljati, prevajalnik vrne napako, odgovornost za eksplicitno navajanje živ\-ljenj\-skih dob pa preda programerju. 

\begin{rust-success}
fn id(x: &i32) -> &i32;
fn id<'a>(x: &'a i32) -> &'a i32;
\end{rust-success}

\subsection{Opombe}
% Non-lexical lifetimes
Od leta 2022 Rust podpira neleksikalne življenjske dobe (angl. non-lexical lifetimes)~\cite{Matsakis_2018, Matsakis_et_al_2022}, pri katerih se preverjanje izposoj in premikov izvaja nad grafom poteka programa (angl. control flow graph) namesto nad abstraktnim sintaktičnim drevesom programa~\cite{weiss2021oxide, StackedBorrows}. Prevajalnik zna s pomočjo neleksikalnih življenjskih dob dokazati, da sta dve zaporedni spremenljivi izposoji varni, če ena izmed njih ni nikjer uporabljena. Na tak način Rust zagotovi bolj drobnozrnat (angl. fine grained) pogled na program, saj prevajalnik vrača napake ob manj veljavnih programih.

Potrebno je poudariti, da bi se vsi primeri v tem poglavju v trenutni različici Rusta zaradi neleksikalnih življenjskih dob vseeno prevedli. Primere smo namreč zaradi boljšega razumevanja in jedrnatosti prikaza nekoliko poenostavili. Ker deklarirane spremenljivke nikjer v prihodnosti niso več uporabljene, zna Rust z analizo živosti prepoznati, da sta npr. dve zaporedni spremenljvi izposoji varni, saj vrednost ni nikoli več spremenjena. Zato za vse primere v tem poglavju predpostavljamo, da so tako premaknjene, kot tudi izposojene spremenljivke v nadaljevanju programa še nekje uporabljene.

\subsection{Sorodno delo}
% Formalizacije Rusta
Kljub temu, da Rust v svoji dokumentaciji~\cite{klabnik2023rust} zagotavlja, da je njegov model upravljanja s pomnilnika varen, pa njegovi razvijalci niso nikoli uradno formalizirali niti njegove operacijske semantike, niti sistema tipov, niti modela za prekrivanje (angl. aliasing model). V literaturi se je tako neodvisno uveljavilo več modelov, ki skušajo čimbolj natančno formalizirati semantiko premikov in izposoj. Model Patina~\cite{reed2015patina} formalizira delovanje analize izposoj v Rustu in dokaže, da velja izrek o varnosti (angl. soundness) za varno (angl. safe) podmnožico jezika Rust. Model RustBelt~\cite{10.1145/3158154} poda prvi formalen in strojno preverjen dokaz o varnosti za jezik, ki predstavlja realistično podmnožico jezika Rust. Model Stacked borrows~\cite{StackedBorrows} definira operacijsko semantiko za dostopanje do pomnilnika v Rustu in model za prekrivanje (angl. aliasing model) in ga strojno dokaže. Model Oxide~\cite{weiss2021oxide} je formalizacija Rustovega sistema tipov in je tudi prva semantika, ki podpira tudi neleksikalne življenjske dobe. 