\section{Linearen sistem tipov}

Večina programskih jezikov uporablja neomejen (angl. unrestricted) sistem tipov, kar omogoča da je lahko vsaka spremenljivka v programu uporabljena poljubno mnogokrat.
% TODO: Manjka

Linearen sistem tipov zahteva, da je vsaka spremenljivka uporabljena \textit{natanko enkrat}. Tak sistem tipov omogoča implementacijo bolj učinkovitega čiščenja pomnilnika, saj zagotavlja, da lahko objekt izbrišemo takoj po uporabi spremenljivke. Poleg linearnih tipov pa taki sistemi tipov običajno določajo tudi neomejene tipe (angl. unrestricted types), ki omogočajo večkratno uporabo spremenljivk, saj se izkaže, da so v večini primerov linearni tipi preveč omejujoči.

Programski jezik Granule~\cite{orchard2019quantitative} je \textit{strong} funkcijski jeizk, ki v svojem sistemu tipov združuje tako linearne, kot tudi (angl. graded modal) tipe. 
Za čiščenje pomnilnika se uporablja avtomatski čistilec.

V članku \cite{10.1145/3649848} programski jezik Granule razširijo s poenostavljenimi pravili za lastništvo in izposoje na podlagi tistih iz Rusta. Za čiščenje pomnilnika graded modal tipov, se še vedno uporablja avtomatski čistilec, medtem ko se za čiščenje unikatnih in linearnih tipov uporablja Rustov model upravljanja s pomnilnikom.

% Kaj je Granule
% Kako skrbi za pomnilnik.
% V Haskell so tudi že dodali linearne tipe
% Kaj so linearni tipi.
% Kaj so graded modal tipi.
% Kaj so eksistenčni tipi.
% Kaj so uniqueness tipi.
% Programski jezik